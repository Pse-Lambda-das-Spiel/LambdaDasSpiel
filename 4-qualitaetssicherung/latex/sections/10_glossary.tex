\section{Glossar}
\begin{description}
	\item[Statische Codeanalyse]
	Statische Codeanalyse ist ein statisches Testverfahren für Software, welches zur Compilerzeit durchgeführt wird. Hierbei wird der Quellcode formaler Prüfungen unterzogen bei welchen Fehler in Form von Code-Qualität oder Formatierung entdeckt werden können.
		
	\item[Dynamische Codeanalyse]
	Dynamische Codeanalyse ist ein dynamisches Testverfahren für Software, welcher zur Laufzeit durchgeführt wird. Hierbei können Programmfehler aufgedeckt werden, welche durch evtl. variierende Parameter oder variierender Nutzer-Interaktion auftreten können.
		
	\item[Tool / Werkzeug]
	Tools bzw. Werkzeuge beschreiben die Mittel,  mit welchen entweder statische oder dynamische Codeanalyse durchgeführt werden können.
\end{description}


\section{Anhang}
Im Anhang befinden sich der detaillierte Coverage-Report, der mithilfe von EMMA erstellt worden ist.