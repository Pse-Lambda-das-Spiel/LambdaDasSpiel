\section{Verbesserungen}
\subsection{\texttt{package lambda.model.shop}}
\begin{itemize}
\item Einfügen weiterer Items
\begin{itemize} 
\item Im Laufe der Qualitätssicherung wurde der Shop mit Items erweitert, sodass nun insgesamt 5 Musik-Items, 2 Background-Items, sowie 2 Elemente-Items erwerbbar sind.
	\end{itemize}
\end{itemize}

\subsection{\texttt{package lambda.model.level}}
\begin{itemize}
\item Neue Exception: InvalidLevelIdException
\begin{itemize} 
\item Im Laufe der Qualitätssicherung wurde das Projekt um eine Exception erweitert, welche geworfen wird, falls man ein Level mit einer invaliden Id beantragt.
	\end{itemize}
\end{itemize}

\begin{itemize}
\item Schwierigkeitsgrade
\begin{itemize} 
\item Die Schwierigkeitsgrade wurden im Laufe der Qualitätssicherung so ausgebaut, dass nun jede Stufe seine eigene Musik, sowie seinen eigenen Hintergrund bereitgestellt bekommt.
	\end{itemize}
\end{itemize}

\begin{itemize}
\item Tutorials
\begin{itemize} 
\item Die Tutorials wurde nochmal komplett überarbeitet und es sind jetzt auch Tutorials im Redktionsmodus möglich.
	\end{itemize}
\end{itemize}

\subsection{\texttt{package lambda.viewcontroller.editor} und \texttt{package lambda.viewcontroller.reduction}}
\begin{itemize}
\item Keine Überflüssigen Buttons und Dialoge
\begin{itemize} 
\item Im Editor- und ReductionViewController werden Buttons und Dialoge, die in dem Level nicht vorkommen oder nicht gebraucht werden entweder nicht angezeigt oder deaktiviert.
	\end{itemize}
\end{itemize}