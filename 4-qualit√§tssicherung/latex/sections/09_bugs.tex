\section{Behobene Bugs}
\subsection{Shop}
\begin{itemize}
\item Anzeige von mehreren aktivierten Items pro Kategorie
\begin{itemize} 
\item Der Bug im Shop, wenn man zwei verschiedene Items einer Kategorie aktivierte, dass beide als aktiviert angezeigt werden, wurde behoben
	und somit wird der Status jedes Items nun richtig angezeigt.
	\end{itemize}
\end{itemize}

\begin{itemize}
\item Fehlende Anzeige aktivierter Items nach Programmneustart
\begin{itemize} 
\item Der Bug im Shop, dass ein bereits aktiviertes Item nach einem Neustart des Programms nicht mehr als aktiviert angezeigt wird, wurde behoben
	und nun werden beim Laden eines Profils, wechseln des Profils sowie beim Neustarten alle Zustände der Items richtig angezeigt.
	\end{itemize}
\end{itemize}

\begin{itemize}
\item Schließen offener Kategorien nach Item-Einkauf
\begin{itemize} 
\item Der Bug im Shop, wenn ein Item gekauft wurde und direkt nach dem Kauf alle offenen Kategorien geschlossen wurden, wurde behoben und
	der aktuelle Zustand einer Kategorie (ob offen oder geschlossen) wird nach einem Kauf nun beibehalten.
	\end{itemize}
\end{itemize}

\begin{itemize}
\item Maske zur Darstellung von Lämmern
\begin{itemize} 
\item Anstatt der Applikationsmaske wird jetzt tatsächlich die Abstraktionsmaske für Lämmer mit Zauberstab verwendet.
	\end{itemize}
\end{itemize}

\begin{itemize}
\item Reduktionsanimationen
\begin{itemize} 
\item Fehlende Animationen fürs Zaubern, Verschwinden und für den Farbwechsel von Lämmern sind jetzt eingebunden.
	\end{itemize}
\end{itemize}

\begin{itemize}
\item Entfernen von Spielelementen mit Attribut \textit{locked}
\begin{itemize} 
\item Solche Spielelemente sind jetzt tatsächlich nicht mehr editierbar und können auch nicht durch Einfügen einer Klammerung darüber entfernt werden.
	\end{itemize}
\end{itemize}

\begin{itemize}
\item Reduktionsgrenzen
\begin{itemize} 
\item Vorwärtsschritte nach Erreichen eines Minimalterms sowie Rückwärtsschritte bei leerem Verlauf sind jetzt nicht mehr möglich.
	\end{itemize}
\end{itemize}

\begin{itemize}
\item Prüfung der Gültigkeit eines leeren Terms
\begin{itemize} 
\item Wirft jetzt keine NullPointerException mehr.
	\end{itemize}
\end{itemize}

\begin{itemize}
\item Pausierung der automatischen Reduktion
\begin{itemize} 
\item Ist jetzt möglich. Der Play-/Pausebutton wird dann auch entsprechend angepasst.
	\end{itemize}
\end{itemize}

\begin{itemize}
\item Deaktivieren der Steuerungsknöpfe im Reduktionsmodus
\begin{itemize} 
\item Solche Knöpfe lassen sich jetzt nur noch Drücken, wenn der Status des Reduktionsmodels dies zulässt.
	\end{itemize}
\end{itemize}

\begin{itemize}
\item Mehrere Zeiger bei Drag\&Drop
\begin{itemize} 
\item Drag\&Drop ist jetzt nur noch mit dem ersten Zeiger möglich.
	\end{itemize}
\end{itemize}