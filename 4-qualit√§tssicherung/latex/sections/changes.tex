\section{Änderungen am Pflichtenheft}

\begin{itemize}
\item Alphakonversion
\begin{itemize} 
\item Ist jetzt schärfer implementiert als im Pflichtenheft und im Spielkonzept von Bret Victor vorgegeben. Ursprünglich wurde eine Alphakonversion vor einer Applikation ausgeführt, wenn im linken und rechten Teilterm eine gemeinsame Farbe verwendet wird. $(\lambda x.xx)(\lambda x.xx)$ ist ein Beispiel, bei dem auf diese Art und Weise eine Konversion unnötig ausgeführt wird. In der aktuellen Version wird deshalb eine Alphakonversion nur in zwei Fällen ausgeführt: Wenn eine freie Variable im rechten Term durch die Applikation gebunden würde $(z.B. (\lambda x. \lambda y.x) y)$ oder wenn nach der Applikation eine Variable durch mehrere Abstraktionen gebunden wäre $(z.B. (\lambda x. \lambda y.x) \lambda y.y)$, wird eine Alphakonversion im linken Teilterm durchgeführt.
	\end{itemize}
\end{itemize}