\section{Testwerkzeuge}
\subsection{Statische Werkzeuge zur Codeanalyse}
\begin{description}
\item[Checkstyle ]Checkstyle ist ein freies statisches Codeanalyse-Werkzeuge, welches verwendet wird, um die Java-Codequalität zu verbessern.
	Dieses statische Analyse-Werkzeug lässt sich mit einer geeigneten Datei konfigurieren, um zu entscheiden nach welchen Kriterien
	die Qualität verbessert wird. Der von uns verwendete Checkstyle ist im Repository vorzufinden und prüft unter anderem die Methodenlänge, 
	Dateilänge, Anzahl an Methodenparameter, richtige Formatierung des Codes (Whitespaces etc.), Namenkonventionen 



\item[FindBugs] FindBugs ist ein freies statisches Codeanalyse-Werkzeug, welches verwendet wird, um insbesondere Fehlermuster in Java-Code bzw. im Java-Bytecode zu finden.
	Diese Fehlermuster können unter anderem auf wirkliche Fehler hindeuten, müssen sie aber nicht.
	Wir haben dieses Werkzeug verwendet, um die eben angesprochenen Fehlermuster aufzudecken und zu überprüfen, ob diese zu Fehlern
	führen oder nicht. Alle gefundenen Bugs waren recht klein und führten demnach nicht zu Fehlern, weshalb eine Beseitigung nicht nötig war.


\subsection{Dynamische Werkzeuge zur Codeanalyse}


\begin{description} 
\item[JUnit] JUnit ist ein Java-Testframework, welches verwendet wird, um Java-Programme zu testen. Dazu werden Tests geschrieben, welche man immer wieder automatisiert durchlaufen lassen.
	Des Weiteren stellt dieses Framework einige Methoden bereit, um Attribute der einzelnen Klassen auf Richtigkeit zu überprüfen.
	Dieses Werkzeug wurde von uns verwendet, um die Zusammenarbeit einzelner Module zu testen, sowie die Überprüfung der Attribute und weiterhin auch, um durch EMMA (siehe nächster Punkt) 
	eine Testüberdeckung veranschaulichen zu können.


\item[Code Coverage mit EMMA] EMMA ist ein dynamische Codeanalyse-Werkzeug, welches verwendet wird, um die Code Coverage (Testüberdeckung) von JUnit-Tests zu messen.
	Mit dieser Testüberdeckung lässt sich oft  Code herausfiltern, welcher niemals ausgeführt wird und weiterhin die Abdeckung von diversen Modultests.
	EMMA basiert auf einer partiellen Zeilenüberdeckung, in der teilweise ausgeführte Zeilen registriert werden, z.B. bei verzweigenden Anweisungen.
	Dieses dynamische Werkzeug wurde von uns verwendet, um die Testüberdeckung der JUnit-Model-Tests zu beobachten und zu analysieren. 

\item[MonkeyRunner] MonkeyRunner ist ein dynamisches Werkzeug, welches mit Python-Programmen ausgeführt wird. Das Werkzeug funktioniert für Android-Geräte oder solche Emulatoren.
	Dieses dynamische Werkzeug führt vorgeschriebene Python-Dateien aus und klickt automatisiert an die in der Python-Datei definierten Stellen, wobei es immer wieder Screenshots aufnimmst.
	MonkeyRunner wurde von uns verwendet, um unsere definierten globalen Testfälle und Systemtests nachzustellen, um diese nicht immer wieder per Hand durchzugehen.


\item[Monkey] Monkey ist ein dynamisches Werkzeug, welches gerne für Belastungs- und Stresstests für Android-Applikationen verwendet wird. Dieses Tool läuft entweder direkt auf Android-Geräten oder auf solchen Emulatoren.
	Dieses dynamische Werkzeug generiert zufällige und hochfrequentierte Eingaben und prüft somit, ob irgendwelche unerwarteten Exceptions oder Fehler auftreten bzw. ab welcher Frequenz von Eingaben die Applikation 	  abstürzt. Dieses Tool wurde von uns verwendet, um die eben beschriebenen Belastungs- und Stresstests durchzuführen.
\end{description}