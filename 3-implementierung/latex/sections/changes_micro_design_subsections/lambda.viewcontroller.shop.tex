\subsection{\texttt{package lambda.viewcontroller.shop}}

\item[Allgemein] \hfill \\ Die Kontrolle über die Status der Klasse $\texttt{\textbf{DropDownMenuViewController}}$, ob diese "geöffnet" oder "geschlossen" im Shop angezeigt wird, wurde der Klasse $\texttt{\textbf{ShopViewController}}$ überschrieben, da diese für die allgemeine Anzeige im Shop zuständig ist und die Buttons für die $\texttt{\textbf{DropDownMenuViewController}}$ hält


\subsubsection{\normalfont \texttt{public class \textbf{ShopItemViewController}<T extends ShopItemModel> extends Actor implements ShopItemModelObserver}}

\begin{description}

\item[Allgemein] \hfill \\ Die Klasse wurde zu einer generischen Klasse umgeschrieben, da es so einfacher und übersichtlicher ist für jeden Item-Typ einen ViewController zu erstellen. Des Weiteren wurde die Klasse um einen TextButton erweitert, der je nach Status des Items angepasst und im Shop angezeigt wird.

\item[Hinzugefügte Methoden] \hfill \\
	\vspace{-.8cm}
	\begin{itemize}
		\item $\texttt{public void \textbf{setCurrentState}()}$ \\ Diese Methode wird aufgerufen, wenn sich der Status eines Items ändert und passt demnach den TextButton dieses Items an, damit visuell der neue Status im Shop signalisiert wird.
\end{itemize}
\end{description}