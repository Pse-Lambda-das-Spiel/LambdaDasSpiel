\subsection{\texttt{package lambda.viewcontroller}}

\begin{description}
	\item[Notizen] \hfill
	\begin{itemize}
		\item Controller in AssetViewController umbenannt.
		\item AssetViewController jetzt in lambda.viewcontroller.assets zu finden.
	\end{itemize}
\end{description}

\subsubsection{\normalfont \texttt{public class \textbf{AudioManager} implements ProfileManagerObserver, SettingsModelObserver}}

\begin{description}
\item[Allgemein] \hfill \\ Neue Klasse zum Verwalten bzw. Abspielen der Geräusche und Musik im Spiel. Wurde eingeführt, um die Tonausgabe zu vereinfachen.
	
\item[Methoden] \hfill \\
	\vspace{-.8cm}
	\begin{itemize}
		\item $\texttt{\textbf{queueAssets}()}$ \\ Wird aufgerufen, um die Sound-Assets und die Standardmusik im Spiel zu laden.
		
		\item $\texttt{\textbf{init}()}$ \\ Initialisiert den AudioManager nachdem \textbf{queueAssets}() aufgerufen wurde und die Assets fertig geladen sind.
				
		\item $\texttt{\textbf{setLoggedIn}()}$ \\ Wechselt den AudioManager in eingeloggten bzw. ausgeloggten Zustand. (Unterschiedliche Tonausgabe in beiden Bereichen)
		
		\item $\texttt{\textbf{playSound}()}$ \\ Wird benutzt um ein Geräusch (z.B. Button-Klick) abzuspielen.
	
		\item $\texttt{\textbf{playDefaultMusic}()}$ \\ Spielt die Standardmusik des Spiels in einer Schleife ab.
	
		\item $\texttt{\textbf{playMusic}()}$ \\ Spielt gegebene Musik in einer Schleife ab.
	\end{itemize}
\end{description}

\subsubsection{\normalfont \texttt{public abstract class \textbf{ViewController} implements Screen, ProfileManagerObserver}}

\begin{description}
\item[Allgemein] \hfill \\ (Umbenennung) Ursprünglich die Controller-Klasse des Entwurfs. Implementiert jetzt das ProfileManagerObserver-Interface, da dies sonst selbst von fast allen Unterklassen implementiert wird.
	
\item[Neue Methoden] \hfill \\
	\vspace{-.8cm}
	\begin{itemize}
		\item $\texttt{\textbf{queueAssets}()}$ \\ Wurde eingeführt, damit beim Ladevorgang jeder Unterklasse von \textbf{ViewController} ihre benötigten Assets laden kann. Die Methode übergibt dabei benötigte Assets in die Warteschlange des AssetManagers.
		\item $\texttt{\textbf{create}()}$ \\ Wird nach dem Ladevorgang aufgerufen, um die ViewController, mit den jetzt geladenen Assets, zu initialisieren.
	\end{itemize}
\end{description}

\subsubsection{\normalfont \texttt{public abstract class \textbf{StageViewController} extends ViewController}}

\begin{description}
\item[Allgemein] \hfill \\ Neue Klasse. Spezialisierter \textbf{ViewController}, welcher einen eigenen Bildschirm im Spiel repräsentiert. \textbf{ViewController}, die einen Bildschirm darstellen, haben viele häufig identische Methoden. Der \textbf{StageViewController} gibt den vom \textbf{ViewController} geerbten Methoden eine solche Standardimplementierung, wodurch Redundanz verhindert wird.

\item[Neue Methoden] \hfill \\
	\vspace{-.8cm}
	\begin{itemize}
		\item $\texttt{\textbf{getStage}()}$ \\ Liefert die Stage, die den kompletten Bildschirm darstellt (auf ihr sind alle GUI-Elemente verankert) zurück.
		\item $\texttt{\textbf{getLastViewController}()}$ \\ Liefert eine Referenz auf den ViewController zurück, der gezeigt werden soll, wenn die Zurücktaste von Android gedrückt wird.
		\item $\texttt{\textbf{setLastViewController}()}$ \\ Setzt den ViewController, der gezeigt werden soll, wenn die Zurücktaste von Android gedrückt wird.
	\end{itemize}
\end{description}