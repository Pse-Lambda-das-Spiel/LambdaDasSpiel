\section{Funktionale Anforderungen}

In den folgenden Kapiteln werden Anforderung, die den Wunschkriterien zugeordnet sind, mit einem "`+"' hinter der Identifikationsnummer markiert.

\subsection{Profile}

Profile ermöglichen das Nutzen des Programms durch mehrere Anwender auf demselben Gerät.

\begin{itemize}
\item /FA110/ Jeder Nutzer soll ein eigenes Profil anlegen können, welches eindeutig durch den Profil-Namen gekennzeichnet ist. 
\item /FA120/ Bei Programmstart wird das zu benutzende Profil ausgewählt oder ein neues Profil erstellt.
\item /FA130/ Die zu einem Profil gespeicherten Daten können nach Erstellen des Profils geändert werden.
\item /FA140/ Ein Profil kann nach dem Erstellen gelöscht werden.
\item /FA150/ Zu jedem Profil werden Spieleinstellungen, Spielfortschritt und Spielstatistik gespeichert.
\item /FA160/ Beim Auswählen eines Profils werden die dazu gespeicherten Daten automatisch geladen.
\item /FA170/ Nach dem Auswählen eines Profils wird der Nutzer durch eine profilspezifische Nachricht begrüßt.
\end{itemize}

\subsection{Spielmodi}

Im Editormodus werden Terme erstellt, angezeigt und bearbeitet.

\begin{itemize}
\item /FA210/ Durch Ziehen können Objekte sowohl von der Werkzeugleiste als auch vom Term ausgewählt werden.
\item /FA220/ Während ein Objekt ausgewählt ist, wird die aktuelle Stelle im Term, an der das Objekt platziert werden kann, farblich markiert.
\item /FA230/ Während ein Objekt ausgewählt ist, kann durch Loslassen des Zeigers das Objekt an der aktuellen Stelle im Term platziert werden.
\item /FA240/ Während ein Objekt ausgewählt ist, kann durch Loslassen des Zeigers über der Werkzeugleiste das Objekt gelöscht werden.
\item /FA250/ Neu hinzugefügte Objekte erhalten die Farbe weiß.
\item /FA260/ Durch Drücken eines Objektes wird ein Dialog-Fenster geöffnet, in dem eine neue Farbe für das Objekt ausgewählt werden kann.
\item /FA270/ Das Ändern der Position oder Farbe von durch ein Level vorgegebenen Objekten kann optional unterdrückt werden.
\item /FA280/ Ein Term wird gültig sobald er den Spielregeln entspricht, worauf man in den Reduktionsmodus wechseln kann.
\end{itemize}

Im Reduktionsmodus werden Terme schrittweise durch Beta-Konversionen reduziert.

\begin{itemize}
\item /FA310/ Konversionen können einzeln schrittweise ausgeführt werden.
\item /FA320/ Konversionen können einzeln schrittweise zurückgesetzt werden.
\item /FA330/ Konversionen können automatisch abgespielt werden bis ein minimaler Term erreicht ist oder der Nutzer das Abspielen beendet.
\end{itemize}

\subsection{Level}

Das Spiel ist in Level eingeteilt, welche vom Spieler nacheinander freigeschaltet und gelöst werden.

\begin{itemize}
\item /FA410/ Das Ziel des Leveltyps "`Eingabe-Bestimmung"' ist es, im Editormodus einen gültigen Term zu erstellen, welcher durch Konversionen im Reduktionsmodus in einen minimalen im Level vorgegebenen Term umgewandelt wird.
\item /FA420/ Das Ziel des Leveltyps "`Ausgabe-Bestimmung"' ist es, im Editormodus einen gültigen Term zu erstellen, welcher aus der Reduktion eines im Level vorgegebenen Terms hervorgeht.
\item /FA430/ Nur das erste Level ist nach Erstellung eines Profils freigeschaltet, durch Abschließen eines Levels wird das darauf folgende Level freigeschaltet.
\item /FA440/ Nach erfolgreichen Abschließen eines Levels hat der Spieler die Möglichkeit ins nächste Level oder ins Hauptmenü zu wechseln.
\item /FA450/ Nach erfolgreichen Abschließen eines Levels wird der Spieler durch ein Nachrichtenfenster über den Erfolg informiert.
\item /FA460/ Nach erstmaligem erfolgreichen Abschließen des letzten Levels wird dem Spieler durch ein Nachrichtenfenster zu seiner Leistung gratuliert.
\item /FA470/ Über das Levelauswahlmenü kann der Spieler seinen Level-Fortschritt beobachten und bereits freigeschaltete Level erneut spielen.
\item /FA480/ Die Level sind in verschiedene Schwierigkeitsstufen eingeteilt, welche durch Farbe und Hintergrundbild voneinander unterscheidbar sind.
\item /FA490+/ Der Spieler kann sich Tipps und Lösungsansätze zum aktuellen Level anzeigen lassen.
\end{itemize}

\subsection{Gamification}

Durch ein Belohnungssystem wird die Spiel- und Lernfreude der Nutzer gesteigert.

\begin{itemize}
\item /FA510+/ Das erstmalige Abschließen eines Levels wird mit einer Level-spezifischen Anzahl von Münzen belohnt.
\item /FA520+/ Im Shop kann der Nutzer gegen Eintausch von Münzen neue Sounds, Hintergrundbilder und Avatare freischalten.
\item /FA530+/ Vor dem Freischalten eines Elements im Shop wird der Spieler um eine Bestätigung gebeten.
\item /FA540+/ Im Shop freigeschaltete Elemente können dort durch ein Kontrollkästchen aktiviert werden. Standardmäßig werden gerade gekaufte Sounds oder Hintergrundbilder sofort aktiviert.
\item /FA550+/ Für bestimmte Leistungen werden Erfolgsnachrichten in einem Erfolgsmenü angezeigt. Folgende Erfolge sind mindestens möglich:
\begin{itemize}
\item Erstes Level geschafft.
\item Alle Level eines Schwierigkeitsgrades erfolgreich abgeschlossen.
\item Alle Level des Spiels erfolgreich abgeschlossen.
\item 5 Level erfolgreich abgeschlossen.
\item 10 Level erfolgreich abgeschlossen.
% TODO
\end{itemize}
\item /FA560+/ Im Ladebildschirm werden zur Unterhaltung des Nutzers Comic-artige Sprechblasen mit lustigen und interessanten Texten angezeigt.
\end{itemize}

\subsection{Eltern und Lehrer}

Statistiken und Optionen für Eltern und Lehrer geben diesen einen Einblick in den Lernfortschritt des Kindes.

\begin{itemize}
\item /FA610/ In einem Statistikmenü werden verschiedene Daten angezeigt. Folgende Daten sind mindestens möglich:
\begin{itemize}
\item Spielzeit
\item Anzahl Versuche gesamt
\item Erfolgsquote für das Bestehen der Level
\item Häufigkeit der Nutzung von Hinweisen
% TODO
\end{itemize}
\item /FA620+/ Über den Lehrermodus wird im Editor- sowie im Reduktionsmodus der aktuelle Lambda-Term in mathematischer Darstellung angezeigt.
\item /FA630+/ Im Farbenblindenmodus, werden Objekte im Spiel mit verschiedenen Muster und Graustufen dargestellt, um das Spielen für Farbenblinde zu ermöglichen.
\end{itemize}

\subsection{Benutzerinteraktion}

\begin{itemize}
\item /FA710/ Über den Touchscreen des Gerätes kann der Nutzer mit dem Programm interagieren.
\item /FA720/ Über den Lautstärkeregler des Gerätes oder im Optionsmenü kann die Lautstärke des Programms verändert werden.
\item /FA730/ Mit dem "`Zurück-Knopf"' des Gerätes kann von einem Menü in das vorherige Menü gewechselt werden.
\item /FA740/ Mit der Drag\&Drop-Geste kann im Editormodus der Term auf dem Bildschirm verschoben werden.
\item /FA750+/ Mit die Pinch-Geste kann im Editormodus die Zoomstufe verändert werden.
\item /FA760/ Über ein Kontrollkästchen im Hauptmenü kann die Hintergrundmusik an- und ausgeschaltet werden.
\item /FA770+/ Das Programm unterstützt das Auswählen der Sprachen Englisch, Deutsch und Französisch.
\end{itemize}
