\section{Zielbestimmung}

	Das Produkt vermittelt Grundschülern auf spielerische Art und Weise das Konzept des untypisierten $\lambda$-Kalküls und dadurch die Grundlage funktionaler Programmierung   			    

\subsection{Musskriterien}

\begin{itemize}
	\item Bedienen über ein Smartphone per Toucheingabe
	\item Auflösen von Lamm-Konstellationen(Analogie zu $\lambda$-Termen)
	\begin{itemize}
		\item Bestimmen eines Endresultats bei einer bereits gegebenen, vollständigen Anordnung von Lämmern 
		\item Vervollständigen einer Anordnung von Lämmern, um ein gegebenes Endresultat zu erreichen
	\end{itemize}
	\item Verfolgen des Lernfortschritts durch Eltern oder Lehrer
	\begin{itemize}
		\item Achievementsystem
		\item Statistik
	\end{itemize}
	\item Aufrechterhaltung der Langzeitmotivation des Spielers
	\item Interaktive und intuitive Einführung zur Erklärung des Spiels und seiner Modi
\end{itemize}

\subsection{Wunschkriterien}

\begin{itemize}
	\item Für Tablets angepasste Version
	\item Spiel als Desktop-Anwendung
	\begin{itemize}
		\item Windows-Anwendung
		\item Mac OS X-Anwendung
	\end{itemize}
	\item Erstellen, Konfigurieren und Löschen von mehreren Spielerprofilen
	\item Anzeigen von Hinweisen zur Lösung des Level-Ziels
	\item Freier Modus, in dem eigene Level erstellt und simuliert werden können
	\item Belohnungen für das erstmalige Lösen der Level
	\item Eintauschen von Belohnungen gegen höherwertige Belohnungen in einem In-Game-Shop
	\item Option für Farbenblinde, durch die Farbenblinde Hilfestellungen beim Spiel bekommen
	\item Lehreroption, durch die die richtigen $\lambda$-Terme beim Abspielen der Level angezeigt werden
	\item Vermittlung einer kindergerechten Hintergrundgeschichte mittels Animationen
	\item Englisch als unterstütze Sprache
	\item Französisch als unterstütze Sprache
	\item Optionale Ausführung von unterschiedlichen Reduktionsstrategien bei der $\beta$-Reduktion von Lamm-Konstellationen (Normalreihenfolge ist die Standard-Reduktionsstrategie)
	\begin{itemize}
		% Kennt jemand davon den deutschen Fachausdruck?
		\item Applicative Order
		\item Call-By-Name
		\item Call-By-Value
	\end{itemize}
		
\end{itemize}

\subsection{Abgrenzungskriterien}

\begin{itemize}
	\item Keine direkte Bezugnahme zur funktionalen Programmierung oder dem $\lambda$-Kalkül (außer bei aktivierter Lehreroption) 
	\item Keine Unterstützung von Online-Funktionen
	\item Kein Mehrspielermodus
\end{itemize}
