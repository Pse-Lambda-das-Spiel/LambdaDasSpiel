\section{Glossar}

\begin{description}
	\item[Achievement] \hfill \\
	Errungenschaft/Erfolg. Bezeichnet eine Auszeichnung für eine bestimmte Leistung.\\
	Kann im Spiel gewonnen werden.\\
	Beispiel: "Lambda für Anfänger" - Du hast das Tutorial erfolgreich abgeschlossen
	
	\item[Android] \hfill \\
	Betriebssystem und Softwareplattform hauptsächlich für mobile Geräte. Das Spiel/Produkt wird in erster Linie 
	für Android entwickelt.
	
	\item[App] \hfill \\
	Kurz für Application oder Applikation und bezeichnet Anwendungssoftware. Im Deutschen meistens jene von mobilen Geräten.
	
	\item[Avatar] \hfill \\
	Im Profil des Benutzers ist der Avatar, ein Bild, aus einer vorgefertigten Sammlung wählbar.
	Der Avatar soll, mit dem Profilnamen, den Nutzer repräsentiert und ist rein kosmetisch.
	Ziel ist dem Spieler die Möglichkeit zu geben sein Profil persönlicher gestalten.
	
	\item[Beta-Konversion] \hfill \\
	Stellt im Lambda-Kalkül das Konzept der Funktionsanwendung dar.
	
	\item[Button] \hfill \\
	Ein Steuerungselement, dass ein Knopf oder eine Taste repräsentiert. Durch einen Klick darauf wird eine Aktion, wie zum Beispiel
	das Schließen des aktuellen Fensters, ausgeführt.
	
	\item[Checkbox] \hfill \\
	Ein Element grafischer Benutzeroberflächen. Wird meist als Kästchen dargestellt, das mit einem Klick aktiviert (abgehakt) oder
	wieder deaktiviert wird. Zum Beispiel um die Musik in einem Spiel an oder aus zu stellen.
	
	\item[Drag\&Drop-Geste] \hfill \\
	Der Benutzer klickt ein Objekt auf dem Bildschirm. Solange er nicht loslässt, "zieht" er das Objekt, wodurch
	es sich typischerweise mit seinem Finger mitbewegt. Lässt er los, lässt er das Objekt wieder "fallen", wodurch
	es an seinem neuen Platz abgelegt wird.
	
	%\item[Dropdown-Menü] \hfill \\
	
	\item[Editor-Modus] \hfill \\
	Ein logischer Modus, in dem der Spieler die Möglichkeit hat Lämmer und Edelsteine in eine bestimmte Anordnung zu bringen.
	
	\item[Image] \hfill \\
	Bezeichnet ein Bild bzw. Grafik im Spiel.
	
	\item[Label] \hfill \\
	Element der Benutzeroberfläche, dass verwendet wird um Text auszugeben. 
	
	\item[Lambda-Kalkül] \hfill \\
	Das Lambda-Kalkül ist eine formale Sprache, die zur Untersuchung von mathematischen Funktionen entwickelt wurde.
	Die Grundlagen des Lambda-Kalküls zu erlernen ist das Ziel des Produkts.
	
	\item[Level] \hfill \\
	Hier bezeichnet ein Level ein abgeschlossenen Teil des Spiels. Der Spieler betritt/startet das Level. Ihm wird eine Aufgabe,
	wie zum Beispiel ein Rätsel gestellt. Nach dem Abschließen der Aufgabe verlässt der Spieler das Level wieder.
	
	\item[Münzen] \hfill \\
	Virtuelle Währung des Spiels. Sie kann zum Beispiel durch erfolgreiches Abschließen eines Levels verdient werden.
	Gewonnene Münzen können im Spiel dann wiederum ausgegeben werden.
	
	\item[Pinch-Geste] \hfill \\
	Durch Berührung zweier Punkte auf dem Bildschirm wird zwischen ihnen ein nicht sichtbares Zentrum erzeugt.
	Bewegt der Benutzer seine Finger näher zum Zentrum oder entfernt er sie weiter davon werden für gewöhnlich
	Aktionen wie die, entsprechend der Bewegung, Vergrößerung oder Verkleinerung von Objekten durchgeführt.
	
	\item[Popup] \hfill \\
	Popups sind kleinere Fenster, die auf dem Bildschirm erscheinen und das Fenster hinter ihnen teilweise verdecken.
	Sie zeigen oft zusätzliche Inhalte an oder suchen Bestätigung für eine Aktion des Nutzers.
	
	\item[Profil] \hfill \\
	Profile machen das Benutzen des Spiels von mehreren Personen möglich. Jeder Benutzer hat ein eigenes Profil,
	dass alle seine Daten (Name, Spielfortschritt usw.) speichert. Der Benutzer wählt beim Start sein Profil aus, wodurch das Spiel,
	wie er es zuvor verlassen hat, geladen wird, obwohl zum Beispiel in der Zwischenzeit ein Zweiter auf einem anderen Profil gespielt hat.
	
	\item[Reduktion] \hfill \\
	Oder Beta-Reduktion. Anderer Name für die Beta-Konversion, falls diese ausschließlich von links nach rechts angewandt wird.
	
	\item[Reduktions-Modus] \hfill \\
	Ein logischer Modus, in dem eine bestimmte gegebene Anordnung von Spielelementen gemäß den Spielregeln umgewandelt wird.
	
	\item[Shop] \hfill \\
	Der Shop ist ein Menü im Spiel, indem die im Spiel existierende Währung der Münzen gegen verschiedenste Dinge eingetauscht werden kann.
	
	\item[Slider] \hfill \\
	Ein Schieberegler. Er besteht aus einer Leiste und einem Zeiger, der auf dieser verschiebbar ist. Durch Verschieben des Zeigers kann aus 
	einem Bereich an Werten ausgewählt werden. Zum Beispiel zur Einstellung der Lautstärke.
	
	\item[Smartphone] \hfill \\
	Ein mobiles Telefon, dass mehr Computer-Funktionalitäten besitzt, als ein herkömmliches Telefon. Häufiges Merkmal ist ein sogenannter
	Touchscreen, der zu einem großen Teil zur Bedienung benutzt wird.
	
	\item[Tablet] \hfill \\
	Es ähnelt einem Smartphone und verwendet häufig auch für Smartphones entwickelte Betriebssysteme. 
	Ein Tablet ist aber normalerweise um ein Vielfaches größer und besitzt dadurch einen größeren Touchscreen.

	\item[Term] \hfill \\
	Mit einem Term wird hier ein Ausdruck im Lambda-Kalkül beschrieben. Das Spiel basiert auf der Idee solche Terme kindgerecht 
	zu visualisieren.
	
	\item[Touchscreen] \hfill \\
	Berührungsempfindlicher Bildschirm. Durch Berührungen und Gesten auf dem Touchscreen kann ein Gerät bedient werden.
	
	\item[Zoom] \hfill \\
	Durch Zoomen scheint Spieler den Bildausschnitt näher zu einem Objekt zu bewegen oder ihn weiter davon zu entfernen.
	Dadurch kann der Nutzer kleine Objekte größer darstellen oder sich bei vielen Objekten einen Überblick "von oben" verschaffen. 
\end{description}
