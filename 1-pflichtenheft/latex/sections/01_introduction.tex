\section{Einleitung}

In unserer heutigen Welt ist das Programmieren und das Entwickeln von Software im Allgemeinen eine immer wichtiger werdende Fähigkeit, da Computer in immer mehr Aspekten unseres Alltags Einzug erhalten. Ein Leben ohne Computer ist dabei schon in vielen Bereichen undenkbar. \newline 
Die funktionale Programmierung ist dabei ein in den letzten Jahren immer wichtiger werdendes Paradigma, was sich anhand der Vorteile bei der Parallelisierung sowie den hohen Abstraktionsgrad erklären lässt. Die Grundlage der funktionalen Programmierung bildet der (untypisierte) Lambda-Kalkül ($\lambda$-Kalkül, der trotz seiner Mächtigkeit nur wenige Regeln besitzt und im Wesentlichen einfach zu verstehen und anzuwenden ist.

Aus diesen Gründen gibt es immer mehr Ansätze, Kinder schon im Grundschulalter mit einigen grundlegenden Programmierkonzepten wie dem $\lambda$-Kalkül vertraut zu machen. So ist ein oft geforderter Ansatz, ein zusätzliches Unterrichtsfach in der Grundschule einzuführen. Dafür hilfreich sind beispielsweise Lernspiele, die Kindern einen spielerischen Einstieg in die Programmierkonzepte vermitteln und den Eltern sowie Lehrern ermöglichen, dabei den Lernfortschritt ihrer Kinder zu verfolgen.\newline
Eine solche Spielidee \footnote{\url{http://worrydream.com/AlligatorEggs/}} zur Erklärung des $\lambda$-Kalküls wurde von Bret Victor veröffentlicht. In diesem Spiel werden die $\lambda$-Terme von verschiedenfarbigen Alligatoren und Eiern dargestellt und die Regeln des Kalküls darauf übertragen. Dies vermeidet die formale Notation des Kalküls, welche Kinder von dem Konzept abschrecken könnte und erklärt ihnen das Konzept auf eine einfache und verständliche Weise. Angelehnt an die Spielidee von Bret Victor soll nun eine Applikation für Android entwickelt werden, die von unserem Team "`Lamb.da"' genannt wird. 

Bei dieser Applikation benutzen wir jedoch eine andere Analogie als Bret Victor, so werden die $\lambda$-Terme durch verschiedenfarbige, zaubernde Lämmer und durch verschiedenfarbige Edelsteine dargestellt. Diese Analogie wurde von uns gewählt, da sie eine motivierende und ansprechende Thematik für Kinder darstellt und da durch sie der Applikationsname zum Wortspiel wird. So ist das "`Lamb"' im Namen "`Lamb.da"' die englische Bezeichnung für Lamm und das "`da"' sind die Anfangsbuchstaben des Edelsteintyps Danburit, während das ganze Wort "`Lambda"' der Name des zu vermittelnden Kalküls ist. 

