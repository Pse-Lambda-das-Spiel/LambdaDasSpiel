\section{Grobentwurf}

\subsection{libGDX}
%kann auch gelöscht werden wenn jemand was Besseres einfällt
libGDX ist ein Framework allein für die Spieleentwicklung und wurde gewählt um die Entwicklung des Produkts zu vereinfachen.
Viele von uns benötigte Funktionalitäten sind in libGDX schon mit inbegriffen, wodurch wir erheblich Implementierungsarbeit einsparen.
Dabei wird es zum Beispiel einfach gemacht die Benutzeroberfläche oder Animationen zu erstellen.
Zusätzlich soll das Produkt auf mehreren Plattformen laufen, sodass uns libGDX erlaubt es möglichst plattformneutral zu entwickeln 
und relativ einfach entsprechende, spezialisierte Programmversionen zu erstellen.

\subsection{Model-View-Controller}
Das Prinzip von Model-View-Controller (MVC) ist allgemein weit verbreitet und heute schon nahezu Standard für Entwurf von Softwaresystemen. 
MVC erlaubt uns ein gut gekapseltes Programm zu erstellen. Dies erleichtert die Implementierung bzw. den Entwurf, da Klassen teils in sich abgeschlossen sind
und eine logische Einheit bilden. Durch MVC und unser dazugehöriges Beobachter-Muster (siehe Observer) erhalten Klassen eine lose Kopplung,
die Flexibilität und auch die einfache Wiederverwertung von Klassen ermöglicht, sollte das Programm oder Programmteile weiterentwickelt werden.
Im Gegensatz zum Standard MVC-Modell haben wir, aufgrund einer gewissen, schon von libGDX vorgegebenen Vermischung von View und Controller,
auf getrennte View und Controller verzichtet. So besteht unser Modell aus Model und "ViewControllern", wie man im Feinentwurf sehen wird.

\subsection{Observer}

\subsection{Visitor}

\subsection{Singleton-Pattern}
Da wir verschiedene Klassen benötigen von denen es aber nur ein Objekt geben soll und ein globaler sowie einfacher Zugriff auf diese Objekte geben soll, bietet sich das Singleton-Pattern natürlich besonders an. So gibt es beispielsweise die Klasse $\texttt{AssetModel}$, welche alle benötigten Ressourcen für die Applikation enthält und von mehreren Klassen gleichzeitig benutzt wird.

\subsection{Strategy}
