\section{Grobentwurf}

\subsection{\texttt{libGDX}}
%kann auch gelöscht werden wenn jemand was Besseres einfällt
libGDX ist ein Framework allein für die Spieleentwicklung und wurde gewählt um die Entwicklung des Produkts zu vereinfachen.
Viele von uns benötigte Funktionalitäten sind in libGDX schon mit inbegriffen, wodurch wir erheblich Arbeit einsparen.
Dabei wird es zum Beispiel besonders einfach gemacht die Benutzeroberfläche oder Animationen zu erstellen.
Zusätzlich soll das Produkt auf mehreren Plattformen laufen. libGDX erlaubt es uns möglichst plattformneutral zu entwickeln 
und relativ einfach entsprechende, spezialisierte Programmversionen zu erstellen.

\subsection{\texttt{Model-View-Controller}}

\subsection{\texttt{Observer}}

\subsection{\texttt{Visitor}}

\subsection{\texttt{Singleton-Pattern}}
Da wir verschiedene Klassen benötigen von denen es aber nur ein Objekt geben soll und ein globaler sowie einfacher Zugriff auf diese Objekte geben soll, bietet sich das Singleton-Pattern natürlich besonders an. So gibt es beispielsweise die Klasse $\texttt{AssetModel}$, welche alle benötigten Ressourcen für die Applikation enthält und von mehreren Klassen gleichzeitig benutzt wird.


\subsection{\texttt{Strategy}}
