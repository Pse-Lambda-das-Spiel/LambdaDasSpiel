\section{Glossar}
\begin{description}
\item[UML]
Die Unified Modeling Language (vereinheitlichte Modellierungssprache) oder kurz UML ist eine Modellierungssprache, also eine Sprache mit der ein zu implementierendes Software-System grafisch beschrieben wird.
UML bietet eine einheitliche Notation, welche man durch etliche Arten von Diagrammen in viele Gebieten anwenden kann.

\item[LibGDX]
LibGDX ist ein auf Java basierendes Framework für die Entwicklung von Multiplattform-Spielen. So erlaubt es mit der gleichen Code-Basis die Entwicklung für Desktop und Mobile Endgeräte wie Windows, Linux, Mac OS X, Android, iOS und HTML5. 

\item[Klassendiagramm]
Klassendiagramme sind Diagramme der Unified Modeling Language (UML) und dienen zur grafischen Beschreibung des Aufbaus und Zusammenspiels von Klassen.
Es beschreibt neben Attributen und Methoden auch weitere Abhängigkeiten wie Oberklassen oder zu implementierende Interfaces.

\item[Sequenzdiagramm]
Sequenzdiagramme beschreiben eine zeitliche Abfolge und Kommunikation zwischen einer Menge von Objekten in einer bestimmten Szene dar. Es beschreibt wie die beteiligten Objekte miteinander kommunizieren und arbeiten, sowie auch Abläufe eines einzelnen Anwendungsfalles.

\item[Entwurfsmuster]
Entwurfsmuster (engl. design pattern) sind eine Art Lösungsschablonen für immer wieder auftretende Entwurfssprobleme. Ein Entwurfsmuster zeichnet sich dadurch aus, dass der damit beschriebene Entwurf oder geschriebene Code vordefiniert strukturiert ist und somit schnell und effizient von Dritten verstanden wird, wenn sie dieses Entwurfsmuster kennen und verstanden haben. Voraussetzung dafür ist, dass man Entwurfsmuster mit bedacht wählt und richtig anwendet, damit es zu einer eleganten Lösung kommt.

\item[Identifizierer]
Ein Identifizierer oder kurz Id ist eine eindeutig und einmalig vergebene Nummer für ein Objekt, um dieses wieder zu erkennen.

\item[JSON]

\end{description}