\subsection{\texttt{package lambda.viewcontroller}}

\subsubsection{\normalfont \texttt{public class \textbf{Controller} implements Screen}}

\begin{description}
\item[Beschreibung] \hfill \\ Eine Oberklasse für alle ViewController, die einen Bildschirm darstellen.

\item[Attribute] \hfill \\
	\vspace{-.8cm}
	\begin{itemize}
		\item $\texttt{LambdaGame \textbf{game}}$ \\ Eine Referenz zur Hauptklasse.
	\end{itemize}
	
\item[Methoden] \hfill \\
	\vspace{-.8cm}
	\begin{itemize}
		\item $\texttt{public void \textbf{setGame}(LambdaGame game)}$ \\ Setzt die Referenz zur Hauptklasse.
		\begin{description}
			\item[Parameter] \hfill \\
			\vspace{-.8cm}
			\begin{itemize}
				\item $\texttt{LambdaGame game}$ \\ Die Referenz zur Hauptklasse
			\end{itemize}
		\end{description}
		
		\item $\texttt{public LambdaGame \textbf{getGame}()}$ \\ Gibt die Referenz zur Hauptklasse zurück.
		\begin{description}
			\item[Rückgabe] \hfill \\
			\vspace{-.8cm}
			\begin{itemize}
				\item Die Referenz zur Haupeklasse.
			\end{itemize}
		\end{description}
				
		\item $\texttt{public void \textbf{dispose}()}$ \\ Wird aufgerufen, wenn der Screen all seine Ressourcen freigeben soll.
		
		\item $\texttt{public void \textbf{show}()}$ \\ Wird automatisch aufgerufen, wenn der Screen als aktueller Screen für das Spiel gesetzt wird.
	
		\item $\texttt{public void \textbf{hide}()}$ \\ Wird automatisch aufgerufen, wenn der Screen nicht mehr der aktuelle Screen des Spiels ist.
	
		\item $\texttt{public void \textbf{resume}()}$ \\ Wird automatisch aufgerufen, wenn die Applikation nach einem pausierten Zustand fortgesetzt wird.	
	
		\item $\texttt{public void \textbf{pause}()}$ \\ Wird automatisch aufgerufen, wenn die Applikation pausiert wird.
	
		\item $\texttt{public void \textbf{render}(float delta)}$ \\ Wird automatisch zum Zeichnen und Darstellen des Screens aufgerufen.
		\begin{description}
			\item[Parameter] \hfill \\
			\vspace{-.8cm}
			\begin{itemize}
				\item $\texttt{float delta}$ \\ Die Zeit in Sekunden seit dem letzten Aufruf dieser Methode.
			\end{itemize}
		\end{description}	
	
		\item $\texttt{public void \textbf{resize}(int width, int height)}$ \\ Wird automatisch aufgerufen, wenn sich die Bildschirmgröße geändert hat.
		\begin{description}
			\item[Parameter] \hfill \\
			\vspace{-.8cm}
			\begin{itemize}
				\item $\texttt{int width}$ \\ Die neue Breite in Pixel.
				\item $\texttt{int height}$ \\ Die neue Höhe in Pixel.
			\end{itemize}
		\end{description}
	\end{itemize}
\end{description}