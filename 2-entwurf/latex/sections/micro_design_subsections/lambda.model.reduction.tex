\subsection{\texttt{package lambda.model.reduction}}

\subsubsection{\normalfont \texttt{public class \textbf{ReductionModel} extends Observable<ReductionModelObserver}}

\begin{description}
\item[Beschreibung] \hfill \\ Führt die vollständige Reduktion eines Lambda Terms aus.

\item[Attribute] \hfill \\
	\vspace{-.8cm}
	\begin{itemize}
		\item $\texttt{private Stack<LambdaRoot> \textbf{history}}$ \\ Speichert alle Lambda Terme vor und nach jedem Reduktionsschritt, sodass solche Schritte wieder rückgängig gemacht werden können.
		\item $\texttt{private boolean \textbf{paused}}$ \\ Gibt an, ob das automatische Reduzieren gerade pausiert ist.
		\item $\texttt{private boolean \textbf{pauseRequested}}$ \\ Gibt an, ob eine Anfrage vorliegt, wonach das automatische Reduzieren pausiert werden soll.
		\item $\texttt{private BetaReductionVisitor \textbf{strategy}}$ \\ Die Strategie, die bei der Reduktion verwendet wird.
		\item $\texttt{private LambdaRoot \textbf{current}}$ \\ Der aktuelle Term.
		\item $\texttt{private boolean \textbf{busy}}$ \\ Gibt an, ob gerade ein Reduktionsschritt ausgeführt wird.
		\item $\texttt{private LevelContext \textbf{context}}$ \\ Enthält alle wichtige Daten für das Spielen des aktuellen Levels.
	\end{itemize}
	
\item[Konstruktoren] \hfill \\
	\vspace{-.8cm}
	\begin{itemize}
		\item $\texttt{public \textbf{ReductionModel}(LambdaRoot term, BetaReductionVisitor strategy, LevelContext context)}$ \\ Instanziiert ein Objekt dieser Klasse mit dem gegebenen Anfangsterm, der Strategie und dem Level-Kontext.
		\begin{description}
			\item[Parameter] \hfill \\
			\vspace{-.8cm}
			\begin{itemize}
				\item $\texttt{LambdaRoot term}$ \\ Der Term, der reduziert werden soll.
				\item $\texttt{BetaReductionVisitor strategy}$ \\ Die Reduktionsstrategie.
				\item $\texttt{LevelContext context}$ \\ Der Level-Kontext.
			\end{itemize}
		\end{description}
	\end{itemize}
	
\item[Methoden] \hfill \\
	\vspace{-.8cm}
	\begin{itemize}
		\item $\texttt{public void \textbf{play}()}$ \\ Startet das automatische Reduzieren, falls das Reduzieren pausiert ist und gerade kein Schritt ausgeführt, indem erst mit Hilfe von $\texttt{setPaused}$ das Pausieren ausgeschaltet wird und dann per $\texttt{step}$ die Schritte ausgeführt werden.
		
		\item $\texttt{public void \textbf{pause}()}$ \\ Setzt $\texttt{pauseRequested}$ auf $\texttt{true}$, falls das Reduzieren nicht bereits pausiert ist.
		
		\item $\texttt{public void \textbf{step}()}$ \\ Tut nichts, wenn gerade ein Schritt ausgeführt wird oder eine Pause-Anfrage vorliegt. Startet sonst einen neuen Thread und setzt dabei erst $\texttt{busy}$ auf $\texttt{true}$ und führt solange Schritte aus, bis eine Pause-Anfrage vorliegt. Setzt am Ende $\texttt{busy}$ auf $\texttt{false}$ und $\texttt{paused}$ auf $\texttt{true}$ mit Hilfe von $\texttt{setBusy}$ und $\texttt{setPaused}$. Ein Schritt wird ausgeführt, indem erst der aktuelle Term kopiert und im Stack gespeichert wird, und dann eine Kopie der aktuellen Strategie zur Wurzel des aktuellen Terms geschickt wird. Falls kein Reduktionsschritt ausgeführt wurde, ist die Reduktion zu Ende. Sendet dann eine entsprechende Nachricht an alle Observer und beendet die Schleife.
		
		\item $\texttt{public void \textbf{stepRevert}()}$ \\ Falls der Stack nicht leer ist und gerade kein Reduktionsschritt ausgeführt wird, ersetzt dann das erste Kind der aktuellen Wurzel durch das erste Kind der Wurzel des obersten Elements auf dem Stack. Ruft vor und nach dem Schritt entsprechend $\texttt{setBusy}$ auf.
		
		\item $\texttt{public void \textbf{setPaused}(boolean paused)}$ \\ Speichert, ob das automatische Reduzieren pausiert ist, und benachrichtigt alle Observer, falls daran eine Änderung vorliegt.
		\begin{description}
			\item[Parameter] \hfill \\
			\vspace{-.8cm}
			\begin{itemize}
				\item $\texttt{boolean paused}$ \\ Gibt an, ob das automatische Reduzieren pausiert ist.
			\end{itemize}
		\end{description}
		
		\item $\texttt{public void \textbf{setBusy}(boolean busy)}$ \\ Speichert, ob gerade ein Reduktionsschritt ausgeführt wird, und benachrichtigt alle Observer, falls daran eine Änderung vorliegt.
		\begin{description}
			\item[Parameter] \hfill \\
			\vspace{-.8cm}
			\begin{itemize}
				\item $\texttt{boolean busy}$ \\ Gibt an, ob gerade ein Reduktionsschritt ausgeführt wird.
			\end{itemize}
		\end{description}
		
		\item $\texttt{public LevelContext \textbf{getLevelContext}()}$ \\ Gibt den Level-Kontext zurück, mit dem diese Reduktion ausgeführt wird.
		\begin{description}
			\item[Rückgabe] \hfill \\
			\vspace{-.8cm}
			\begin{itemize}
				\item Der Level-Kontext, mit dem diese Reduktion ausgeführt wird.
			\end{itemize}
		\end{description}
	\end{itemize}
\end{description}

\subsubsection{\normalfont \texttt{public interface \textbf{ReductionModelObserver}}}

\begin{description}
\item[Beschreibung] \hfill \\ Stellt Methoden zur Verfügung, die von einem $\texttt{ReductionModel}$ als Nachrichten an seine Observer aufgerufen werden.

\item[Methoden] \hfill \\
	\vspace{-.8cm}
	\begin{itemize}
		\item $\texttt{public void \textbf{pauseChanged}(boolean paused)}$ \\ Aufgerufen, falls das automatische Reduzieren pausiert oder fortgesetzt wird.
		\begin{description}
			\item[Parameter] \hfill \\
			\vspace{-.8cm}
			\begin{itemize}
				\item $\texttt{boolean paused}$ \\ Gibt an, ob das automatische Reduzieren pausiert ist.
			\end{itemize}
		\end{description}
		
		\item $\texttt{public void \textbf{busyChanged}(boolean busy)}$ \\ Aufgerufen, falls sich der Zustand, ob gerade ein Reduktionsschritt ausgeführt wird, ändert.
		\begin{description}
			\item[Parameter] \hfill \\
			\vspace{-.8cm}
			\begin{itemize}
				\item $\texttt{boolean paused}$ \\ Gibt an, ob gerade ein Reduktionsschritt ausgeführt wird.
			\end{itemize}
		\end{description}
		
		\item $\texttt{public void \textbf{reductionFinished}(boolean levelComplete)}$ \\ Aufgerufen, wenn die Reduktion abgeschlossen ist.
		\begin{description}
			\item[Parameter] \hfill \\
			\vspace{-.8cm}
			\begin{itemize}
				\item $\texttt{boolean levelComplete}$ \\ Gibt an, ob der finale Term mit dem Level-Ziel übereinstimmt.
			\end{itemize}
		\end{description}
	\end{itemize}
\end{description}
