\subsection{\texttt{package lambda.viewcontroller.reduction}}

\subsubsection{\normalfont \texttt{public class \textbf{ReductionViewController} extends Controller implements ReductionModelObserver}}

\begin{description}
\item[Beschreibung] \hfill \\ Der ViewController zum Reduktions-Modus.

\item[Attribute] \hfill \\
	\vspace{-.8cm}
	\begin{itemize}
		\item $\texttt{ReductionModel \textbf{model}}$ \\ Die Daten und Logik einer Reduktion.
		\item $\texttt{scene2d.Stage \textbf{stage}}$ \\ Enthält alle anzuzeigenden Elemente.
		\item $\texttt{LambdaTermViewController \textbf{lambdaTerm}}$ \\ Zeigt und animiert den aktuellen Lambda Term. Initialisiert mit $\texttt{null}$.
		\item $\texttt{InputMultiplexer \textbf{inputProcesser}}$ \\ Nimmt UI-Events entgegen und behandelt diese.
	\end{itemize}
	
\item[Konstruktoren] \hfill \\
	\vspace{-.8cm}
	\begin{itemize}
		\item $\texttt{public \textbf{ReductionViewController}()}$ \\ Instanziiert ein Objekt dieser Klasse und initialisiert dabei $\texttt{stage}$, $\texttt{inputProcessor}$ und alle UI-Elemente.
	\end{itemize}
	
\item[Methoden] \hfill \\
	\vspace{-.8cm}
	\begin{itemize}		
		\item $\texttt{public void \textbf{update}(ReductionModel model)}$ \\ Aktualisiert den ViewController, sodass dieser das gegebene Reduktions-Model anzeigt.
		\begin{description}
			\item[Parameter] \hfill \\
			\vspace{-.8cm}
			\begin{itemize}
				\item $\texttt{ReductionModel model}$ \\ Das neue Reduktions-Model.
			\end{itemize}
		\end{description}
		
		\item $\texttt{public void \textbf{pauseChanged}(boolean paused)}$ \\ Aufgerufen, falls das automatische Reduzieren pausiert oder fortgesetzt wird. Passt entsprechend die Reduktions-Buttons an.
		\begin{description}
			\item[Parameter] \hfill \\
			\vspace{-.8cm}
			\begin{itemize}
				\item $\texttt{boolean paused}$ \\ Gibt an, ob das automatische Reduzieren pausiert ist.
			\end{itemize}
		\end{description}
		
		\item $\texttt{public void \textbf{busyChanged}(boolean busy)}$ \\ Aufgerufen, falls sich der Zustand, ob gerade ein Reduktionsschritt ausgeführt wird, ändert. Passt entsprechend die Reduktions-Buttons an.
		\begin{description}
			\item[Parameter] \hfill \\
			\vspace{-.8cm}
			\begin{itemize}
				\item $\texttt{boolean paused}$ \\ Gibt an, ob gerade ein Reduktionsschritt ausgeführt wird.
			\end{itemize}
		\end{description}
		
		\item $\texttt{public void \textbf{reductionFinished}(boolean levelComplete)}$ \\ Aufgerufen, wenn die Reduktion abgeschlossen ist. Zeigt den Levelabschluss-Dialog an.
		\begin{description}
			\item[Parameter] \hfill \\
			\vspace{-.8cm}
			\begin{itemize}
				\item $\texttt{boolean levelComplete}$ \\ Gibt an, ob der finale Term mit dem Level-Ziel übereinstimmt.
			\end{itemize}
		\end{description}
				
		\item $\texttt{public void \textbf{dispose}()}$ \\ Wird aufgerufen, wenn der Screen all seine Ressourcen freigeben soll.
		
		\item $\texttt{public void \textbf{show}()}$ \\ Wird automatisch aufgerufen, wenn der Screen als aktueller Screen für das Spiel gesetzt wird.
	
		\item $\texttt{public void \textbf{hide}()}$ \\ Wird automatisch aufgerufen, wenn der Screen nicht mehr der aktuelle Screen des Spiels ist.
	
		\item $\texttt{public void \textbf{resume}()}$ \\ Wird automatisch aufgerufen, wenn die Applikation nach einem pausierten Zustand fortgesetzt wird.	
	
		\item $\texttt{public void \textbf{pause}()}$ \\ Wird automatisch aufgerufen, wenn die Applikation pausiert wird.
	
		\item $\texttt{public void \textbf{render}(float delta)}$ \\ Wird automatisch zum Zeichnen und Darstellen des Screens aufgerufen.
		\begin{description}
			\item[Parameter] \hfill \\
			\vspace{-.8cm}
			\begin{itemize}
				\item $\texttt{float delta}$ \\ Die Zeit in Sekunden seit dem letzten Aufruf dieser Methode.
			\end{itemize}
		\end{description}	
	
		\item $\texttt{public void \textbf{resize}(int width, int height)}$ \\ Wird automatisch aufgerufen, wenn sich die Bildschirmgröße geändert hat.
		\begin{description}
			\item[Parameter] \hfill \\
			\vspace{-.8cm}
			\begin{itemize}
				\item $\texttt{int width}$ \\ Die neue Breite in Pixel.
				\item $\texttt{int height}$ \\ Die neue Höhe in Pixel.
			\end{itemize}
		\end{description}
	\end{itemize}
\end{description}