\subsection{\texttt{package lambda.model.lambdaterm.visitor}}

\subsubsection{\normalfont \texttt{public interface \textbf{LambdaTermVisitor<R>}}}

\begin{description}
\item[Beschreibung] \hfill \\ Repräsentiert einen Besucher auf einer Lambda-Term Baumstruktur. Der Besucher kann Operationen an der Datenstruktur ausführen und hat optional einen Rückgabewert.

\item[Typ-Parameter] \hfill \\
	\vspace{-.8cm}
	\begin{itemize}
		\item $\texttt{<R>}$ \\ Der Typ des Rückgabewertes.
	\end{itemize}

\item[Methoden] \hfill \\
	\vspace{-.8cm}
	\begin{itemize}
		\item $\texttt{public void \textbf{visit}(LambdaRoot node)}$ \\ Besucht die gegebene Wurzel.
		\begin{description}
			\item[Parameter] \hfill \\
			\vspace{-.8cm}
			\begin{itemize}
				\item $\texttt{LambdaRoot node}$ \\ Die besuchte Wurzel. Ist nie $\texttt{null}$.
			\end{itemize}
		\end{description}
			
		\item $\texttt{public void \textbf{visit}(LambdaApplication node)}$ \\ Besucht die gegebene Applikation.
		\begin{description}
			\item[Parameter] \hfill \\
			\vspace{-.8cm}
			\begin{itemize}
				\item $\texttt{LambdaApplication node}$ \\ Die besuchte Applikation. Ist nie $\texttt{null}$.
			\end{itemize}
		\end{description}
		
		\item $\texttt{public void \textbf{visit}(LambdaAbstraction node)}$ \\ Besucht die gegebene Abstraktion.
		\begin{description}
			\item[Parameter] \hfill \\
			\vspace{-.8cm}
			\begin{itemize}
				\item $\texttt{LambdaAbstraction node}$ \\ Die besuchte Abstraktion. Ist nie $\texttt{null}$.
			\end{itemize}
		\end{description}
		
		\item $\texttt{public void \textbf{visit}(LambdaVariable node)}$ \\ Besucht die gegebene Variable.
		\begin{description}
			\item[Parameter] \hfill \\
			\vspace{-.8cm}
			\begin{itemize}
				\item $\texttt{LambdaVariable node}$ \\ Die besuchte Variable. Ist nie $\texttt{null}$.
			\end{itemize}
		\end{description}
		
		\item $\texttt{public R \textbf{getResult}()}$ \\ Gibt das Resultat der Besucheroperation zurück. Wird nur nach einem Besuch ausgeführt. Gibt in der Standard-Implementierung $\texttt{null}$ zurück.
		\begin{description}
			\item[Rückgabe] \hfill \\
			\vspace{-.8cm}
			\begin{itemize}
				\item Das Resultat der Besucheroperation.
			\end{itemize}
		\end{description}
	\end{itemize}
\end{description}

\subsubsection{\normalfont \texttt{public class \textbf{AlphaConversionVisitor} implements LambdaTermVisitor<Object>}}

\begin{description}
\item[Beschreibung] \hfill \\ Repräsentiert einen Besucher auf einer Lambda-Term Baumstruktur, welcher eine Alpha-Konversion auf ihr ausführt.

\item[Attribute] \hfill \\
	\vspace{-.8cm}
	\begin{itemize}
		\item $\texttt{private Color \textbf{old}}$ \\ Die zu ersetzende Farbe.
		\item $\texttt{private Color \textbf{new}}$ \\ Die neue Farbe.
	\end{itemize}

\item[Konstruktoren] \hfill \\
	\vspace{-.8cm}
	\begin{itemize}
		\item $\texttt{public \textbf{AlphaConversionVisitor}(Color old, Color new)}$ \\ instanziiert ein Objekt dieser Klasse mit der gegebenen ersetzten und ersetzenden Farbe.
		\begin{description}
			\item[Parameter] \hfill \\
			\vspace{-.8cm}
			\begin{itemize}
				\item $\texttt{Color old}$ \\ Die zu ersetzende Farbe.
				\item $\texttt{Color new}$ \\ Die neue Farbe.
			\end{itemize}
		\end{description}
	\end{itemize}

\item[Methoden] \hfill \\
	\vspace{-.8cm}
	\begin{itemize}
		\item $\texttt{public void \textbf{visit}(LambdaRoot node)}$ \\ Besucht die gegebene Wurzel und traversiert wenn möglich weiter zum Kindknoten.
		\begin{description}
			\item[Parameter] \hfill \\
			\vspace{-.8cm}
			\begin{itemize}
				\item $\texttt{LambdaRoot node}$ \\ Die besuchte Wurzel.
			\end{itemize}
		\end{description}
			
		\item $\texttt{public void \textbf{visit}(LambdaApplication node)}$ \\ Besucht die gegebene Applikation und traversiert wenn möglich weiter zu beiden Kindknoten.
		\begin{description}
			\item[Parameter] \hfill \\
			\vspace{-.8cm}
			\begin{itemize}
				\item $\texttt{LambdaApplication node}$ \\ Die besuchte Applikation.
			\end{itemize}
		\end{description}
		
		\item $\texttt{public void \textbf{visit}(LambdaAbstraction node)}$ \\ Besucht die gegebene Abstraktion. Dabei wird die Farbe wenn nötig ersetzt und wenn möglich weiter zum Kindknoten traversiert.
		\begin{description}
			\item[Parameter] \hfill \\
			\vspace{-.8cm}
			\begin{itemize}
				\item $\texttt{LambdaAbstraction node}$ \\ Die besuchte Abstraktion.
			\end{itemize}
		\end{description}
		
		\item $\texttt{public void \textbf{visit}(LambdaVariable node)}$ \\ Besucht die gegebene Variable und ersetzt die Farbe wenn nötig.
		\begin{description}
			\item[Parameter] \hfill \\
			\vspace{-.8cm}
			\begin{itemize}
				\item $\texttt{LambdaVariable node}$ \\ Die besuchte Variable.
			\end{itemize}
		\end{description}
	\end{itemize}
\end{description}

\subsubsection{\normalfont \texttt{public class \textbf{ColorCollectionVisitor} implements LambdaTermVisitor<Set<Color>{}>}}

\begin{description}
\item[Beschreibung] \hfill \\ Repräsentiert einen Besucher auf einer Lambda-Term Baumstruktur, der die Menge der benutzten Farben in diesem Term zurückgibt.

\item[Attribute] \hfill \\
	\vspace{-.8cm}
	\begin{itemize}
		\item $\texttt{private Set<Color> \textbf{result}}$ \\ Die Menge aller benutzten Farben.
	\end{itemize}

\item[Konstruktoren] \hfill \\
	\vspace{-.8cm}
	\begin{itemize}
		\item $\texttt{public \textbf{ColorCollectionVisitor}()}$ \\ instanziiert ein Objekt dieser Klasse.
	\end{itemize}

\item[Methoden] \hfill \\
	\vspace{-.8cm}
	\begin{itemize}
		\item $\texttt{public void \textbf{visit}(LambdaRoot node)}$ \\ Besucht die gegebene Wurzel und traversiert wenn möglich weiter zum Kindknoten.
		\begin{description}
			\item[Parameter] \hfill \\
			\vspace{-.8cm}
			\begin{itemize}
				\item $\texttt{LambdaRoot node}$ \\ Die besuchte Wurzel.
			\end{itemize}
		\end{description}
				
		\item $\texttt{public void \textbf{visit}(LambdaApplication node)}$ \\ Besucht die gegebene Applikation und traversiert wenn möglich weiter zu beiden Kindknoten.
		\begin{description}
			\item[Parameter] \hfill \\
			\vspace{-.8cm}
			\begin{itemize}
				\item $\texttt{LambdaApplication node}$ \\ Die besuchte Applikation.
			\end{itemize}
		\end{description}
		
		\item $\texttt{public void \textbf{visit}(LambdaAbstraction node)}$ \\ Besucht die gegebene Abstraktion. Dabei wird die Farbe zur Menge hinzugefügt und wenn möglich weiter zum Kindknoten traversiert.
		\begin{description}
			\item[Parameter] \hfill \\
			\vspace{-.8cm}
			\begin{itemize}
				\item $\texttt{LambdaAbstraction node}$ \\ Die besuchte Abstraktion.
			\end{itemize}
		\end{description}
		
		\item $\texttt{public void \textbf{visit}(LambdaVariable node)}$ \\ Besucht die gegebene Variable und fügt die Farbe zur Menge hinzu.
		\begin{description}
			\item[Parameter] \hfill \\
			\vspace{-.8cm}
			\begin{itemize}
				\item $\texttt{LambdaVariable node}$ \\ Die besuchte Variable.
			\end{itemize}
		\end{description}
		
		\item $\texttt{public Set<Color> \textbf{getResult}()}$ \\ Gibt die Menge der Farben zurück, die in dem besuchten Term benutzt werden.
		\begin{description}
			\item[Rückgabe] \hfill \\
			\vspace{-.8cm}
			\begin{itemize}
				\item Die Menge der benutzten Farben.
			\end{itemize}
		\end{description}
	\end{itemize}
\end{description}

\subsubsection{\normalfont \texttt{public class \textbf{IsColorBoundVisitor} implements LambdaTermVisitor<Boolean>}}

\begin{description}
\item[Beschreibung] \hfill \\ Repräsentiert einen Besucher auf einer Lambda-Term Baumstruktur, der zurückgibt, ob eine Variable mit der gegebenen Farbe in diesem Term gebunden ist.

\item[Attribute] \hfill \\
	\vspace{-.8cm}
	\begin{itemize}
		\item $\texttt{private Color \textbf{color}}$ \\ Die zu überprüfende Farbe.
		\item $\texttt{private boolean \textbf{result}}$ \\ Der Rückgabewert des Besuchs.
	\end{itemize}

\item[Konstruktoren] \hfill \\
	\vspace{-.8cm}
	\begin{itemize}
		\item $\texttt{public \textbf{IsColorBoundVisitor}(Color color)}$ \\ instanziiert ein Objekt dieser Klasse mit der zu überprüfenden Farbe.
		\begin{description}
			\item[Parameter] \hfill \\
			\vspace{-.8cm}
			\begin{itemize}
				\item $\texttt{Color color}$ \\ Die zu überprüfende Farbe.
			\end{itemize}
			\item[Exceptions] \hfill \\
			\vspace{-.8cm}
			\begin{itemize}
				\item $\texttt{NullPointerException}$ \\ Falls $\texttt{color == null}$ ist.
			\end{itemize}
		\end{description}
	\end{itemize}

\item[Methoden] \hfill \\
	\vspace{-.8cm}
	\begin{itemize}
		\item $\texttt{public void \textbf{visit}(LambdaRoot node)}$ \\ Besucht die gegebene Wurzel und beendet die Traversierung hier.
		\begin{description}
			\item[Parameter] \hfill \\
			\vspace{-.8cm}
			\begin{itemize}
				\item $\texttt{LambdaRoot node}$ \\ Die besuchte Wurzel.
			\end{itemize}
		\end{description}
				
		\item $\texttt{public void \textbf{visit}(LambdaApplication node)}$ \\ Besucht die gegebene Applikation und traversiert wenn möglich weiter zum Elternknoten.
		\begin{description}
			\item[Parameter] \hfill \\
			\vspace{-.8cm}
			\begin{itemize}
				\item $\texttt{LambdaApplication node}$ \\ Die besuchte Applikation.
			\end{itemize}
		\end{description}
		
		\item $\texttt{public void \textbf{visit}(LambdaAbstraction node)}$ \\ Besucht die gegebene Abstraktion und überprüft, ob die Farbe hier gebunden ist. Traversiert wenn nötig und möglich weiter zum Elternknoten.
		\begin{description}
			\item[Parameter] \hfill \\
			\vspace{-.8cm}
			\begin{itemize}
				\item $\texttt{LambdaAbstraction node}$ \\ Die besuchte Abstraktion.
			\end{itemize}
		\end{description}
		
		\item $\texttt{public void \textbf{visit}(LambdaVariable node)}$ \\ Besucht die gegebene Variable und traversiert weiter zum Elternknoten.
		\begin{description}
			\item[Parameter] \hfill \\
			\vspace{-.8cm}
			\begin{itemize}
				\item $\texttt{LambdaVariable node}$ \\ Die besuchte Variable.
			\end{itemize}
		\end{description}
		
		\item $\texttt{public Boolean \textbf{getResult}()}$ \\ Gibt zurück, ob die Variable mit der gegebenen Farbe im Term gebunden ist.
		\begin{description}
			\item[Rückgabe] \hfill \\
			\vspace{-.8cm}
			\begin{itemize}
				\item Gibt zurück, ob die Variable mit der gegebenen Farbe gebunden ist.
			\end{itemize}
		\end{description}
	\end{itemize}
\end{description}

\subsubsection{\normalfont \texttt{public class \textbf{ApplicationVisitor} implements LambdaTermVisitor<LambdaTerm>}}

\begin{description}
\item[Beschreibung] \hfill \\ Repräsentiert einen Besucher auf einer Lambda-Term Baumstruktur, welcher eine Applikation ausführt.

\item[Attribute] \hfill \\
	\vspace{-.8cm}
	\begin{itemize}
		\item $\texttt{private Color \textbf{color}}$ \\ Die Farbe der zu ersetzenden Variablen.
		\item $\texttt{private LambdaTerm \textbf{applicant}}$ \\ Das Argument der Applikation.
		\item $\texttt{private LambdaTerm \textbf{result}}$ \\ Der Term nach der Applikation.
		\item $\texttt{private boolean \textbf{hasCheckedAlphaConversion}}$ \\ Initialisiert mit $\texttt{false}$. Speichert, ob bereits überprüft wurde, ob eine Alpha-Konversion vor der Applikation notwendig ist.
	\end{itemize}

\item[Konstruktoren] \hfill \\
	\vspace{-.8cm}
	\begin{itemize}
		\item $\texttt{public \textbf{ApplicationVisitor}(Color color, LambdaTerm applicant)}$ \\ instanziiert ein Objekt dieser Klasse mit der gegebenen Variablenfarbe und dem gegebenen Argument.
		\begin{description}
			\item[Parameter] \hfill \\
			\vspace{-.8cm}
			\begin{itemize}
				\item $\texttt{Color color}$ \\ Die Farbe der zu ersetzenden Variablen.
				\item $\texttt{LambdaTerm applicant}$ \\ Das Argument der Applikation.
			\end{itemize}
			\item[Exceptions] \hfill \\
			\vspace{-.8cm}
			\begin{itemize}
				\item $\texttt{NullPointerException}$ \\ Falls $\texttt{color == null}$ oder $\texttt{applicant == null}$ ist.
			\end{itemize}
		\end{description}
	\end{itemize}

\item[Methoden] \hfill \\
	\vspace{-.8cm}
	\begin{itemize}
		\item $\texttt{public void \textbf{visit}(LambdaRoot node)}$ \\ Kann nie aufgerufen werden, da der besuchte Knoten keinen Elternknoten hat, von wo aus eine Applikation ausgeführt werden könnte.
		\begin{description}
			\item[Parameter] \hfill \\
			\vspace{-.8cm}
			\begin{itemize}
				\item $\texttt{LambdaRoot node}$ \\ Die besuchte Wurzel.
			\end{itemize}
		\end{description}
				
		\item $\texttt{public void \textbf{visit}(LambdaApplication node)}$ \\ Besucht die gegebene Applikation und traversiert weiter zu beiden Kindknoten. Dabei werden die Kindknoten auf die Rückgabewerte beider Besuche gesetzt. Speichert als Rückgabewert den besuchten Term.
		\begin{description}
			\item[Parameter] \hfill \\
			\vspace{-.8cm}
			\begin{itemize}
				\item $\texttt{LambdaApplication node}$ \\ Die besuchte Applikation.
			\end{itemize}
		\end{description}
		
		\item $\texttt{public void \textbf{visit}(LambdaAbstraction node)}$ \\ Besucht die gegebene Abstraktion und traversiert weiter zum Kindknoten. Dabei wird der Kindknoten auf den Rückgabewert des Besuchs gesetzt. Speichert als Rückgabewert den besuchten Term.
		\begin{description}
			\item[Parameter] \hfill \\
			\vspace{-.8cm}
			\begin{itemize}
				\item $\texttt{LambdaAbstraction node}$ \\ Die besuchte Abstraktion.
			\end{itemize}
		\end{description}
		
		\item $\texttt{public void \textbf{visit}(LambdaVariable node)}$ \\ Besucht die gegebene Variable und speichert wenn nötig als Rückgabewert $\texttt{applicant}$.
		\begin{description}
			\item[Parameter] \hfill \\
			\vspace{-.8cm}
			\begin{itemize}
				\item $\texttt{LambdaVariable node}$ \\ Die besuchte Variable.
			\end{itemize}
		\end{description}
		
		\item $\texttt{public LambdaTerm \textbf{getResult}()}$ \\ Gibt den Term nach der Applikation zurück.
		\begin{description}
			\item[Rückgabe] \hfill \\
			\vspace{-.8cm}
			\begin{itemize}
				\item Der besuchte Term.
			\end{itemize}
		\end{description}
		
		\item $\texttt{private void \textbf{checkAlphaConversion}()}$ \\ Überprüft, ob eine Alpha-Konversion notwendig ist, falls dies noch nicht getan wurde, und führt diese wenn nötig aus. Entfernt danach das Argument der Applikation aus dem LambdaTerm.
	\end{itemize}
\end{description}

\subsubsection{\normalfont \texttt{public class \textbf{CopyVisitor} implements LambdaTermVisitor<LambdaTerm>}}

\begin{description}
\item[Beschreibung] \hfill \\ Repräsentiert einen Besucher auf einer Lambda-Term Baumstruktur, welcher die Datenstruktur kopiert und die Kopie zurückgibt.

\item[Attribute] \hfill \\
	\vspace{-.8cm}
	\begin{itemize}
		\item $\texttt{private LambdaTerm \textbf{result}}$ \\ Die Kopie.
	\end{itemize}

\item[Konstruktoren] \hfill \\
	\vspace{-.8cm}
	\begin{itemize}
		\item $\texttt{public \textbf{CopyVisitor}()}$ \\ instanziiert ein Objekt dieser Klasse.
	\end{itemize}

\item[Methoden] \hfill \\
	\vspace{-.8cm}
	\begin{itemize}
		\item $\texttt{public void \textbf{visit}(LambdaRoot node)}$ \\ Besucht die gegebene Wurzel und erstellt eine Kopie. Traversiert zum Kindknoten und speichert den Rückgabewert dieses Besuchs im Kindknoten der Kopie.
		\begin{description}
			\item[Parameter] \hfill \\
			\vspace{-.8cm}
			\begin{itemize}
				\item $\texttt{LambdaRoot node}$ \\ Die besuchte Wurzel.
			\end{itemize}
		\end{description}
				
		\item $\texttt{public void \textbf{visit}(LambdaApplication node)}$ \\ Besucht die gegebene Applikation und erstellt eine Kopie. Traversiert zu beiden Kindknoten und speichert die Rückgabewerte dieser Besuche in den Kindknoten der Kopie.
		\begin{description}
			\item[Parameter] \hfill \\
			\vspace{-.8cm}
			\begin{itemize}
				\item $\texttt{LambdaApplication node}$ \\ Die besuchte Applikation.
			\end{itemize}
		\end{description}
		
		\item $\texttt{public void \textbf{visit}(LambdaAbstraction node)}$ \\ Besucht die gegebene Abstraktion und erstellt eine Kopie. Traversiert zum Kindknoten und speichert den Rückgabewert dieses Besuchs im Kindknoten der Kopie.
		\begin{description}
			\item[Parameter] \hfill \\
			\vspace{-.8cm}
			\begin{itemize}
				\item $\texttt{LambdaAbstraction node}$ \\ Die besuchte Abstraktion.
			\end{itemize}
		\end{description}
		
		\item $\texttt{public void \textbf{visit}(LambdaVariable node)}$ \\ Besucht die gegebene Variable und speichert als Rückgabewert eine Kopie dieser Variable.
		\begin{description}
			\item[Parameter] \hfill \\
			\vspace{-.8cm}
			\begin{itemize}
				\item $\texttt{LambdaVariable node}$ \\ Die besuchte Variable.
			\end{itemize}
		\end{description}
		
		\item $\texttt{public LambdaTerm \textbf{getResult}()}$ \\ Gibt die Kopie zurück.
		\begin{description}
			\item[Rückgabe] \hfill \\
			\vspace{-.8cm}
			\begin{itemize}
				\item Die Kopie.
			\end{itemize}
		\end{description}
	\end{itemize}
\end{description}

\subsubsection{\normalfont \texttt{public class \textbf{RemoveTermVisitor} implements LambdaTermVisitor<Object>}}

\begin{description}
\item[Beschreibung] \hfill \\ Repräsentiert einen Besucher auf einer Lambda-Term Baumstruktur, welcher den besuchten Term aus der Datenstruktur entfernt.

\item[Attribute] \hfill \\
	\vspace{-.8cm}
	\begin{itemize}
		\item $\texttt{private LambdaTerm \textbf{removed}}$ \\ Der zu entfernende Term. Initialisiert mit $\texttt{null}$.
	\end{itemize}

\item[Konstruktoren] \hfill \\
	\vspace{-.8cm}
	\begin{itemize}
		\item $\texttt{public \textbf{RemoveTermVisitor}()}$ \\ instanziiert ein Objekt dieser Klasse.
	\end{itemize}

\item[Methoden] \hfill \\
	\vspace{-.8cm}
	\begin{itemize}
		\item $\texttt{public void \textbf{visit}(LambdaRoot node)}$ \\ Falls ein zu entfernender Term - Kindknoten der Wurzel - gespeichert ist, setzte den Kindknoten auf $\texttt{null}$.
		\begin{description}
			\item[Parameter] \hfill \\
			\vspace{-.8cm}
			\begin{itemize}
				\item $\texttt{LambdaRoot node}$ \\ Die besuchte Wurzel.
			\end{itemize}
		\end{description}
				
		\item $\texttt{public void \textbf{visit}(LambdaApplication node)}$ \\ Besucht die gegebene Applikation. Falls noch kein zu entfernender Term gespeichert ist, speichere diese Applikation und traversiere zum Elternknoten, falls dieser nicht $\texttt{null}$ ist. Ansonsten ist der Term bereits aus der Baumstruktur entfernt. Falls ein zu entfernender Term - Kindknoten in der Applikation - gespeichert ist, ersetze diesen durch $\texttt{null}$.
		\begin{description}
			\item[Parameter] \hfill \\
			\vspace{-.8cm}
			\begin{itemize}
				\item $\texttt{LambdaApplication node}$ \\ Die besuchte Applikation.
			\end{itemize}
		\end{description}
		
		\item $\texttt{public void \textbf{visit}(LambdaAbstraction node)}$ \\ Besucht die gegebene Abstraktion. Falls noch kein zu entfernender Term gespeichert ist, speichere diese Abstraktion und traversiere zum Elternknoten, falls dieser nicht $\texttt{null}$ ist. Ansonsten ist der Term bereits aus der Baumstruktur entfernt. Falls ein zu entfernender Term - Kindknoten der Abstraktion - gespeichert ist, ersetze diesen durch $\texttt{null}$.
		\begin{description}
			\item[Parameter] \hfill \\
			\vspace{-.8cm}
			\begin{itemize}
				\item $\texttt{LambdaAbstraction node}$ \\ Die besuchte Abstraktion.
			\end{itemize}
		\end{description}
		
		\item $\texttt{public void \textbf{visit}(LambdaVariable node)}$ \\ Speichere die Variable als zu entfernenden Term und traversiere zum Elternknoten, falls dieser nicht $\texttt{null}$ ist. Ansonsten ist der Term bereits aus der Baumstruktur entfernt.
		\begin{description}
			\item[Parameter] \hfill \\
			\vspace{-.8cm}
			\begin{itemize}
				\item $\texttt{LambdaVariable node}$ \\ Die besuchte Variable.
			\end{itemize}
		\end{description}
	\end{itemize}
\end{description}

\subsubsection{\normalfont \texttt{public abstract class \textbf{BetaReductionVisitor} implements LambdaTermVisitor<LambdaTerm>}}

\begin{description}
\item[Beschreibung] \hfill \\ Repräsentiert einen Besucher auf einer Lambda-Term Baumstruktur, der eine einzelne Beta-Reduktion gemäß einer Reduktionsstrategie durchführt. Dabei sind Strategien durch Unterklassen dieses Besuchers gegeben.

\item[Attribute] \hfill \\
	\vspace{-.8cm}
	\begin{itemize}
		\item $\texttt{protected LambdaTerm \textbf{result}}$ \\ Der Term nach der Beta-Reduktion.
		\item $\texttt{protected boolean \textbf{hasReduced}}$ \\ Speichert, ob von diesem Besucher bereits eine Reduktion durchgeführt wurde. Initialisiert mit $\texttt{false}$.
		\item $\texttt{protected LambdaTerm \textbf{applicant}}$ \\ Falls der Elternknoten des aktuell besuchten Knotens eine Applikation ist, speichert diese Variable das Argument der Applikation. Initialisiert mit $\texttt{null}$.
	\end{itemize}

\item[Konstruktoren] \hfill \\
	\vspace{-.8cm}
	\begin{itemize}
		\item $\texttt{public \textbf{BetaReductionVisitor}()}$ \\ instanziiert ein Objekt dieser Klasse.
	\end{itemize}

\item[Methoden] \hfill \\
	\vspace{-.8cm}
	\begin{itemize}
		\item $\texttt{public void \textbf{visit}(LambdaRoot node)}$ \\ Traversiere weiter zum Kindknoten und setze diesen auf das Resultat des Besuchs. Speichere als Rückgabewert die besuchte Wurzel.
		\begin{description}
			\item[Parameter] \hfill \\
			\vspace{-.8cm}
			\begin{itemize}
				\item $\texttt{LambdaRoot node}$ \\ Die besuchte Wurzel.
			\end{itemize}
		\end{description}
				
		\item $\texttt{public abstract void \textbf{visit}(LambdaApplication node)}$ \\ Implementiert von der Reduktionsstrategie. Führt entsprechende Operationen zur Reduktion am Term aus $($siehe Unterklassen$)$. Gibt in der Standard-Implementierung nur den besuchten Knoten zurück.
		\begin{description}
			\item[Parameter] \hfill \\
			\vspace{-.8cm}
			\begin{itemize}
				\item $\texttt{LambdaApplication node}$ \\ Die besuchte Applikation.
			\end{itemize}
		\end{description}
		
		\item $\texttt{public abstract void \textbf{visit}(LambdaAbstraction node)}$ \\ Implementiert von der Reduktionsstrategie. Führt entsprechende Operationen zur Reduktion am Term aus $($siehe Unterklassen$)$. Gibt in der Standard-Implementierung nur den besuchten Knoten zurück.
		\begin{description}
			\item[Parameter] \hfill \\
			\vspace{-.8cm}
			\begin{itemize}
				\item $\texttt{LambdaAbstraction node}$ \\ Die besuchte Abstraktion.
			\end{itemize}
		\end{description}
		
		\item $\texttt{public abstract void \textbf{visit}(LambdaVariable node)}$ \\ Implementiert von der Reduktionsstrategie. Führt entsprechende Operationen zur Reduktion am Term aus $($siehe Unterklassen$)$. Gibt in der Standard-Implementierung nur den besuchten Knoten zurück.
		\begin{description}
			\item[Parameter] \hfill \\
			\vspace{-.8cm}
			\begin{itemize}
				\item $\texttt{LambdaVariable node}$ \\ Die besuchte Variable.
			\end{itemize}
		\end{description}
		
		\item $\texttt{public LambdaTerm \textbf{getResult}()}$ \\ Gibt das Resultat der Reduktion zurück.
		\begin{description}
			\item[Rückgabe] \hfill \\
			\vspace{-.8cm}
			\begin{itemize}
				\item Der reduzierte Term.
			\end{itemize}
		\end{description}
	\end{itemize}
\end{description}

