\subsection{\texttt{package lambda.model.shop}}

\subsubsection{\normalfont \texttt{public class \textbf{ShopModel}}}

\begin{description}
\item[Beschreibung] \hfill \\ Repräsentiert einen Shop, der es dem Benutzer ermöglicht seine erspielten Münzen gegen Belohnungen einzutauschen.

\item[Attribute] \hfill \\
	\vspace{-.8cm}
	\begin{itemize}
		\item $\texttt{private ShopItemTypeModel<Music> \textbf{musics}}$ \\ Stellt im Shop die zu kaufenden Items des Typs $\texttt{Music}$ dar.
		\item $\texttt{private ShopItemTypeModel<Image> \textbf{images}}$ \\ Stellt im Shop die zu kaufenden Items des Typs $\texttt{Image}$ dar.		
		\item $\texttt{private ShopItemTypeModel<ElementUIContextFamily> \textbf{elementUIContextFamilies}}$ \\ Stellt im Shop die zu kaufenden Items des Typs $\texttt{ElementUIContextFamily}$ dar.
		\end{itemize}
	
\item[Konstruktoren] \hfill \\
	\vspace{-.8cm}
	\begin{itemize}
		\item $\texttt{public \textbf{ShopModel}()}$ \\ instanziiert ein Objekt dieser Klasse.
	\end{itemize}
	
\item[Methoden] \hfill \\
	\vspace{-.8cm}
	\begin{itemize}
		\item $\texttt{public ShopItemTypeModel<Music> \textbf{getMusics}()}$ \\ Gibt das Attribut $\texttt{musics}$ zurück.
		\begin{description}
			\item[Rückgabe] \hfill \\
			\vspace{-.8cm}
			\begin{itemize}
				\item Gibt $\texttt{musics}$ zurück.
			\end{itemize}
			\end{description}
		
		\item $\texttt{public ShopItemTypeModel<Image> \textbf{getImages}()}$ \\ Gibt das Attribut $\texttt{images}$ zurück.
		\begin{description}
			\item[Rückgabe] \hfill \\
			\vspace{-.8cm}
			\begin{itemize}
				\item Gibt $\texttt{images}$ zurück.
			\end{itemize}
			\end{description}
			
		\item $\texttt{public ShopItemTypeModel<ElementUIContextFamily> \textbf{getElementUIContextFamily}()}$ \\ Gibt das Attribut $\texttt{elementUIContextFamilies}$ zurück.
		\begin{description}
			\item[Rückgabe] \hfill \\
			\vspace{-.8cm}
			\begin{itemize}
				\item Gibt $\texttt{elementUIContextFamilies}$ zurück.
			\end{itemize}
			\end{description}

		\end{itemize}
	\end{description}











\subsubsection{\normalfont \texttt{public abstract class \textbf{ShopItemModel}}}

\begin{description}
\item[Beschreibung] \hfill \\ Repräsentiert ein allgemeines Item, welches man im Shop erwerben kann.

\item[Attribute] \hfill \\
	\vspace{-.8cm}
	\begin{itemize}
		\item $\texttt{private final String \textbf{ID}}$ \\ Gibt den eindeutigen Bezeichner des Items an.
		\item $\texttt{private int \textbf{price}}$ \\ Gibt an für wie viele Münzen dieses Item erworben werden kann.
		\item $\texttt{private ShopModel \textbf{shop}}$ \\ Hält die Referenz auf das ShopModel.
		\item $\texttt{private ShopItemTypeModel \textbf{shopItemType}}$ \\ 	Hält eine Referenz auf ein ShopItemTypeModel, um das aktivierte Item zu setzen.
		\item $\texttt{private boolean \textbf{purchased}}$ \\ Gibt an, ob das Item erworben wurde oder nicht und wird mit $\texttt{false}$ initialisiert. Bei dem Wert $\texttt{true}$ wurde das Item bereits erworben, bei $\texttt{false}$ nicht.
		\end{itemize}
	
\item[Konstruktoren] \hfill \\
	\vspace{-.8cm}
	\begin{itemize}
		\item $\texttt{public \textbf{ShopItemModel}()}$ \\ instanziiert ein Objekt dieser Klasse.

	\end{itemize}
	
\item[Methoden] \hfill \\
	\vspace{-.8cm}
	\begin{itemize}
		\item $\texttt{public String \textbf{getID}()}$ \\ Gibt den Bezeichner des Items zurück.
		\begin{description}
			\item[Rückgabe] \hfill \\
			\vspace{-.8cm}
			\begin{itemize}
				\item Gibt $\texttt{ID}$ zurück.
			\end{itemize}
			\end{description}
		
		\item $\texttt{public int \textbf{getPRICE}()}$ \\ Gibt zurück für welche Anzahl an Münzen dieses Item erworben werden kann.
		\begin{description}
			\item[Rückgabe] \hfill \\
			\vspace{-.8cm}
			\begin{itemize}
				\item Gibt $\texttt{PRICE}$ zurück.
			\end{itemize}
			\end{description}
			
		\item $\texttt{public int \textbf{getShop}()}$ \\ Gibt das ShopModel zurück.
		\begin{description}
			\item[Rückgabe] \hfill \\
			\vspace{-.8cm}
			\begin{itemize}
				\item Gibt $\texttt{shop}$ zurück.
			\end{itemize}
			\end{description}
			
		\item $\texttt{public int \textbf{getShopItemType}()}$ \\ Gibt das ShopItemTypeModel zurück.
		\begin{description}
			\item[Rückgabe] \hfill \\
			\vspace{-.8cm}
			\begin{itemize}
				\item Gibt $\texttt{shopItemType}$ zurück.
			\end{itemize}
			\end{description}
			
		\item $\texttt{public int \textbf{getPurchased}()}$ \\ Gibt über den boolean zurück, ob das Item erworben wurde oder nicht (bei $\texttt{true}$ erworben, bei $\texttt{false}$ noch nicht erworben).
		
			
		\item $\texttt{public void \textbf{buy}()}$ \\ Vergleicht die Anzahl der Münzen des Benutzerprofils mit $\texttt{price}$ und falls diese Anzahl größer oder gleich groß ist, wird $\texttt{purchased}$ von $\texttt{false}$ auf $\texttt{true}$ gesetzt.


			
		\item $\texttt{public void \textbf{activate}()}$ \\ Prüft, ob das Item schon erworben wurde $\texttt{(purchased == true)}$ und setzt dann, insofern dies geschehen ist, das Item als aktuell aktiviertes Item in der entsprechenden Kategorie.
		\begin{description}
		\item[Rückgabe] \hfill \\
			\vspace{-.8cm}
			\begin{itemize}
				\item Gibt $\texttt{purchased}$ zurück.
			\end{itemize}

		\end{description}
	\end{itemize}
\end{description}


\subsubsection{\normalfont \texttt{public class \textbf{MusicModel} extends ShopItemModel}}

\begin{description}
\item[Beschreibung] \hfill \\ Repräsentiert ein Musik-Item, welches im Shop erworben werden kann.

\item[Attribute] \hfill \\
	\vspace{-.8cm}
	\begin{itemize}
		\item $\texttt{private Music \textbf{music}}$ \\ Enthält ein Musikstück, welches im Shop erworben werden kann.

		\end{itemize}
	
\item[Konstruktoren] \hfill \\
	\vspace{-.8cm}
	\begin{itemize}
		\item $\texttt{public \textbf{MusicModel}()}$ \\ instanziiert ein Objekt dieser Klasse.

	\end{itemize}
	
\item[Methoden] \hfill \\
	\vspace{-.8cm}
	\begin{itemize}
		\item $\texttt{public Music \textbf{getMusic}()}$ \\ Gibt das Musik-File zurück.
		\begin{description}
			\item[Rückgabe] \hfill \\
			\vspace{-.8cm}
			\begin{itemize}
				\item Gibt $\texttt{music}$ zurück.
			\end{itemize}
			\end{description}
		
	\end{itemize}
\end{description}

\subsubsection{\normalfont \texttt{public class \textbf{BackgroundImageModel} extends ShopItemModel}}

\begin{description}
\item[Beschreibung] \hfill \\ Repräsentiert ein Hintergrundbild-Item, welches im Shop erworben werden kann.

\item[Attribute] \hfill \\
	\vspace{-.8cm}
	\begin{itemize}
		\item $\texttt{private Music \textbf{music}}$ \\ Enthält ein Hintergrundbild, welches im Shop erworben werden kann.

		\end{itemize}
	
\item[Konstruktoren] \hfill \\
	\vspace{-.8cm}
	\begin{itemize}
		\item $\texttt{public \textbf{ImageModel}()}$ \\ instanziiert ein Objekt dieser Klasse.

	\end{itemize}
	
\item[Methoden] \hfill \\
	\vspace{-.8cm}
	\begin{itemize}
		\item $\texttt{public Image \textbf{getBackgroundImage}()}$ \\ Gibt das Image-File zurück.
		\begin{description}
			\item[Rückgabe] \hfill \\
			\vspace{-.8cm}
			\begin{itemize}
				\item Gibt $\texttt{image}$ zurück.
			\end{itemize}
			\end{description}
		
	\end{itemize}
\end{description}


\subsubsection{\normalfont \texttt{public class \textbf{SpriteModel} extends ShopItemModel}}

\begin{description}
\item[Beschreibung] \hfill \\ Repräsentiert ein Sprite-Item, welches ein Teil eines ElementUIContextFamily-Objektes ist, welches im Shop erworben werden kann.

\item[Attribute] \hfill \\
	\vspace{-.8cm}
	\begin{itemize}
		\item $\texttt{private Sprite \textbf{sprite}}$ \\ Enthält ein Sprite, welches Teil eines ElementUIConext-Objektes ist.

		\end{itemize}
	
\item[Konstruktoren] \hfill \\
	\vspace{-.8cm}
	\begin{itemize}
		\item $\texttt{public \textbf{ImageModel}()}$ \\ instanziiert ein Objekt dieser Klasse.

	\end{itemize}
	
\item[Methoden] \hfill \\
	\vspace{-.8cm}
	\begin{itemize}
		\item $\texttt{public Image \textbf{getSprite}()}$ \\ Gibt das Sprite-File zurück.
		\begin{description}
			\item[Rückgabe] \hfill \\
			\vspace{-.8cm}
			\begin{itemize}
				\item Gibt $\texttt{sprite}$ zurück.
			\end{itemize}
			\end{description}
		
	\end{itemize}
\end{description}




\subsubsection{\normalfont \texttt{public class \textbf{ShopItemTypeModel<T>}}}

\begin{description}
\item[Beschreibung] \hfill \\ Repräsentiert eine ganze Kategorie von erwerbbaren Items.

\item[Typ-Parameter] \hfill \\
	\vspace{-.8cm}
	\begin{itemize}
		\item $\texttt{<Tr>}$ \\ Der Typ einer Kategorie.
	\end{itemize}

\item[Attribute] \hfill \\
	\vspace{-.8cm}
	\begin{itemize}
		\item $\texttt{private String \textbf{typeName}}$ \\ Gibt den Namen des Item-Typs an.
		\item $\texttt{private List<T> \textbf{items}}$ \\ Enthält alle Items vom Typ des Typ-Parameters.
		\item $\texttt{private T \textbf{activatedItem}}$ \\ Enthält das aktuell aktivierte Item vom Typ des Typ-Parameters.
		\end{itemize}
	
\item[Konstruktoren] \hfill \\
	\vspace{-.8cm}
	\begin{itemize}
		\item $\texttt{public \textbf{ShopItemTypeModel}()}$ \\ instanziiert ein Objekt dieser Klasse.

	\end{itemize}
	
\item[Methoden] \hfill \\
	\vspace{-.8cm}
	\begin{itemize}
		\item $\texttt{public String \textbf{getTypeName}()}$ \\ Gibt den Namen der Kategorie zurück.
		\begin{description}
			\item[Rückgabe] \hfill \\
			\vspace{-.8cm}
			\begin{itemize}
				\item Gibt $\texttt{typeName}$ zurück.
			\end{itemize}
			\end{description}
		
		\item $\texttt{public List<T> \textbf{getItems}()}$ \\ Gibt die Liste zurück, welche alle Items des gesetzten Typ-Parameters enthält.
		\begin{description}
			\item[Rückgabe] \hfill \\
			\vspace{-.8cm}
			\begin{itemize}
				\item Gibt $\texttt{items}$ zurück.
			\end{itemize}
			\end{description}
			
		\item $\texttt{public T \textbf{getActivatedItems}()}$ \\ Gibt das Item vom gesetzten Typ zurück, welches aktuell aktiviert ist.
		\begin{description}
			\item[Rückgabe] \hfill \\
			\vspace{-.8cm}
			\begin{itemize}
				\item Gibt $\texttt{activatedItem}$ zurück.
			\end{itemize}
			\end{description}

	\end{itemize}
\end{description}

