\subsection{\texttt{package lambda.viewcontroller.lambdaterm.visitor}}

\subsubsection{\normalfont \texttt{public class \textbf{ViewInsertionVisitor} implements LambdaTermVisitor<Object>}}

\begin{description}
\item[Beschreibung] \hfill \\ Repräsentiert einen Besucher auf einer Lambda-Term Baumstruktur, welcher rekursiv View-Knoten eines gegebenen Lambda-Terms erstellt und in eine gegebenen Lambda-Term ViewController einfügt. Dabei traversiert der Besucher so lange nach oben, bis ein Elternknoten gefunden ist, zu dem ein View-Knoten im Lambda-Term ViewController existiert. Dort wird ein neuer View-Kindknoten erstellt und eingefügt.

\item[Attribute] \hfill \\
	\vspace{-.8cm}
	\begin{itemize}
		\item $\texttt{private LambdaTerm \textbf{inserted}}$ \\ Der Lambda-Term, zu dem View-Knoten erstellt werden.
		\item $\texttt{private LambdaTermViewController \textbf{viewController}}$ \\ Der Lambda-Term ViewController, in den die erstellen View-Knoten eingefügt werden.
		\item $\texttt{private LambdaTerm \textbf{rightSibling}}$ \\ Der Knoten rechts neben dem eingefügten Knoten, falls der Elternknoten eine Applikation ist. Initialisiert mit $\texttt{null.}$
		\item $\texttt{private LambdaTerm \textbf{lastVisited}}$ \\ 
	\end{itemize}

\item[Konstruktoren] \hfill \\
	\vspace{-.8cm}
	\begin{itemize}
		\item $\texttt{public \textbf{ApplicationVisitor}(Color color, LambdaTerm applicant)}$ \\ instanziiert ein Objekt dieser Klasse mit der gegebenen Variablenfarbe und dem gegebenen Argument.
		\begin{description}
			\item[Parameter] \hfill \\
			\vspace{-.8cm}
			\begin{itemize}
				\item $\texttt{Color color}$ \\ Die Farbe der zu ersetzenden Variablen.
				\item $\texttt{LambdaTerm applicant}$ \\ Das Argument der Applikation.
			\end{itemize}
			\item[Exceptions] \hfill \\
			\vspace{-.8cm}
			\begin{itemize}
				\item $\texttt{NullPointerException}$ \\ Falls $\texttt{color == null}$ oder $\texttt{applicant == null}$ ist.
			\end{itemize}
		\end{description}
	\end{itemize}

\item[Methoden] \hfill \\
	\vspace{-.8cm}
	\begin{itemize}
		\item $\texttt{public void \textbf{visit}(LambdaRoot node)}$ \\ Kann nie aufgerufen werden, da der besuchte Knoten keinen Elternknoten hat, von wo aus eine Applikation ausgeführt werden könnte.
		\begin{description}
			\item[Parameter] \hfill \\
			\vspace{-.8cm}
			\begin{itemize}
				\item $\texttt{LambdaRoot node}$ \\ Die besuchte Wurzel.
			\end{itemize}
		\end{description}
				
		\item $\texttt{public void \textbf{visit}(LambdaApplication node)}$ \\ Besucht die gegebene Applikation und traversiert weiter zu beiden Kindknoten. Dabei werden die Kindknoten auf die Rückgabewerte beider Besuche gesetzt. Speichert als Rückgabewert den besuchten Term.
		\begin{description}
			\item[Parameter] \hfill \\
			\vspace{-.8cm}
			\begin{itemize}
				\item $\texttt{LambdaApplication node}$ \\ Die besuchte Applikation.
			\end{itemize}
		\end{description}
		
		\item $\texttt{public void \textbf{visit}(LambdaAbstraction node)}$ \\ Besucht die gegebene Abstraktion und traversiert weiter zum Kindknoten. Dabei wird der Kindknoten auf den Rückgabewert des Besuchs gesetzt. Speichert als Rückgabewert den besuchten Term.
		\begin{description}
			\item[Parameter] \hfill \\
			\vspace{-.8cm}
			\begin{itemize}
				\item $\texttt{LambdaAbstraction node}$ \\ Die besuchte Abstraktion.
			\end{itemize}
		\end{description}
		
		\item $\texttt{public void \textbf{visit}(LambdaVariable node)}$ \\ Besucht die gegebene Variable und speichert wenn nötig als Rückgabewert $\texttt{applicant}$.
		\begin{description}
			\item[Parameter] \hfill \\
			\vspace{-.8cm}
			\begin{itemize}
				\item $\texttt{LambdaVariable node}$ \\ Die besuchte Variable.
			\end{itemize}
		\end{description}
		
		\item $\texttt{public LambdaTerm \textbf{getResult}()}$ \\ Gibt den Term nach der Applikation zurück.
		\begin{description}
			\item[Rückgabe] \hfill \\
			\vspace{-.8cm}
			\begin{itemize}
				\item Der besuchte Term.
			\end{itemize}
		\end{description}
		
		\item $\texttt{private void \textbf{checkAlphaConversion}()}$ \\ Überprüft, ob eine Alpha-Konversion notwendig ist, falls dies noch nicht getan wurde, und führt diese wenn nötig aus. Entfernt danach das Argument der Applikation aus dem LambdaTerm.
	\end{itemize}
\end{description}
