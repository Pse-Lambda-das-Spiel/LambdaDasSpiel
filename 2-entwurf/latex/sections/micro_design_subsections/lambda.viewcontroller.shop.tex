\subsection{\texttt{package lambda.viewcontroller.shop}}

\subsubsection{\normalfont \texttt{public class \textbf{ShopViewController} extends scene2d.Actor}}

\begin{description}
\item[Beschreibung] \hfill \\ Kontrolliert und regelt die Darstellung des Shopmenüs und damit der einzelnen erwerbbaren Items und die Benutzerinteraktion mit dem Menü.

\item[Attribute] \hfill \\
	\vspace{-.8cm}
	\begin{itemize}	
		\item $\texttt{private ShopModel \textbf{model}}$ \\ 
		\item $\texttt{private scene2d.Stage \textbf{stage}}$ \\ 2D-Scene-Graph, der die Hierarchie der gesamten grafischen Komponenten (Akteure mit Typ $\texttt{scene2d.Actor}$ des Screens (der aktuell angezeigte Bildschirm) enthält. 
		\item $\texttt{private InputMultiplexer \textbf{inputProcessor}}$ \\ Delegiert die Eingabe-Ereignisse an die geordnete Liste der InputProcessor, die die Ereignisse empfangen und weiterverarbeiten.
		\item $\texttt{private List<DropDownMenuViewController> \textbf{items}}$ \\ 
		\end{itemize}
	
\item[Konstruktoren] \hfill \\
	\vspace{-.8cm}
	\begin{itemize}
		\item $\texttt{public \textbf{ShopViewController}()}$ \\ instanziiert ein Objekt dieser Klasse.

	\end{itemize}
	
\item[Methoden] \hfill \\
	\vspace{-.8cm}
	\begin{itemize}
		\item $\texttt{public List<DropDownMenuViewController> \textbf{getItems}()}$ \\ Gibt eine Liste mit DropDownViewControllern zurück, welche für den Shop benötigt werden, um die Items anzuzeigen.
		\begin{description}
			\item[Rückgabe] \hfill \\
			\vspace{-.8cm}
			\begin{itemize}
				\item Gibt $\texttt{items}$ zurück.
			\end{itemize}
			\end{description}
		
		\item $\texttt{public void \textbf{update}()}$ \\.


		\end{itemize}
	\end{description}
	
	
\subsubsection{\normalfont \texttt{public class \textbf{ShopItemViewController} extends Controller implements ShopItemModelObserver}}

\begin{description}
\item[Beschreibung] \hfill \\ Kontrolliert und regelt die Darstellung des Items innerhalb des Shops und damit die Benutzerinteraktion mit dem Item.
\item[Attribute] \hfill \\
	\vspace{-.8cm}
	\begin{itemize}	
		\item $\texttt{private ShopItemModel \textbf{model}}$ \\ 
		\item $\texttt{private scene2d.Stage \textbf{stage}}$ \\ 2D-Scene-Graph, der die Hierarchie der gesamten grafischen Komponenten (Akteure mit Typ $\texttt{scene2d.Actor}$ des Screens (der aktuell angezeigte Bildschirm) enthält. 
		\item $\texttt{private InputMultiplexer \textbf{inputProcessor}}$ \\ Delegiert die Eingabe-Ereignisse an die geordnete Liste der InputProcessor, die die Ereignisse empfangen und weiterverarbeiten.
		\end{itemize}
	
\item[Konstruktoren] \hfill \\
	\vspace{-.8cm}
	\begin{itemize}
		\item $\texttt{public \textbf{ShopItemViewController}()}$ \\ instanziiert ein Objekt dieser Klasse.

	\end{itemize}
	
\item[Methoden] \hfill \\
	\vspace{-.8cm}
	\begin{itemize}
		\item $\texttt{public void \textbf{purchasedChanged}(boolean purchased)}$ \\ Aktualisiert die Anzeige eines Items im Shop nach einem erfolgreichen Erwerb oder Profilwechsel.	
		\begin{description}
			\item[Parameter] \hfill \\
			\vspace{-.8cm}
			\begin{itemize}
				\item $\texttt{boolean purchased}$ \\ Gibt an, ob das Item erworben wird oder nach einem Profilwechsel, ob dieses Profil das Item nicht erworben hat.
			\end{itemize}
			\end{description}
		
		\item $\texttt{public void \textbf{activatedChanged}(boolean activated)}$ \\ Aktualisiert die Anzeige eines Items im Shop nach Änderung des aktivierten Items oder Profilwechsel.
		\begin{description}
			\item[Parameter] \hfill \\
			\vspace{-.8cm}
			\begin{itemize}
				\item $\texttt{boolean activated}$ \\ Gibt an, ob das Item aktiviert oder deaktiviert wird bzw. nach einem Profilwechsel, ob dieses aktiviert oder nicht aktiviert ist.
			\end{itemize}
			\end{description}
			
		\item $\texttt{public void \textbf{dispose}()}$ \\ Wird aufgerufen, wenn der Screen all seine Ressourcen freigeben soll.
		
		\item $\texttt{public void \textbf{show}()}$ \\ Wird automatisch aufgerufen, wenn der Screen als aktueller Screen für das Spiel gesetzt wird.
	
		\item $\texttt{public void \textbf{hide}()}$ \\ Wird automatisch aufgerufen, wenn der Screen nicht mehr der aktuelle Screen des Spiels ist.
	
		\item $\texttt{public void \textbf{resume}()}$ \\ Wird automatisch aufgerufen, wenn die Applikation nach einem pausierten Zustand fortgesetzt wird.	
	
		\item $\texttt{public void \textbf{pause}()}$ \\ Wird automatisch aufgerufen, wenn die Applikation pausiert wird.
	
		\item $\texttt{public void \textbf{render}(float delta)}$ \\ Wird automatisch zum Zeichnen und Darstellen des Screens aufgerufen.
			
		\item $\texttt{public void \textbf{draw}()}$ \\ Aktualisiert die Anzeige eines Items im Shop nach Änderung des aktivierten Items oder Profilwechsel.


		\end{itemize}
	\end{description}
	
