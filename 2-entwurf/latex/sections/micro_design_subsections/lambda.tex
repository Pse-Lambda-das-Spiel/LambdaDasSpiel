\subsection{\texttt{package lambda}}

\subsubsection{\normalfont \texttt{public class \textbf{Observable}<Observer>}}

\begin{description}
\item[Beschreibung] \hfill \\ Repräsentiert ein Objekt, das von Beobachtern überwacht werden kann. Dabei informiert das Objekt alle Beobachter, sobald Änderungen an ihm vorgenommen werden.

\item[Typ-Parameter] \hfill \\
	\vspace{-.8cm}
	\begin{itemize}
		\item $\texttt{<Observer>}$ \\ Der Typ eines Beobachters.
	\end{itemize}

\item[Attribute] \hfill \\
	\vspace{-.8cm}
	\begin{itemize}
		\item $\texttt{private List<Observer> \textbf{observers}}$ \\ Die Liste der Beobachter dieses Objektes.
	\end{itemize}
	
\item[Konstruktoren] \hfill \\
	\vspace{-.8cm}
	\begin{itemize}
		\item $\texttt{public \textbf{Observable}()}$ \\ instanziiert ein Objekt dieser Klasse.
	\end{itemize}
	
\item[Methoden] \hfill \\
	\vspace{-.8cm}
	\begin{itemize}
		\item $\texttt{public void \textbf{addObserver}(Observer o)}$ \\ Fügt den gegebenen Beobachter diesem Objekt hinzu, sodass dieser bei Änderungen informiert wird.
		\begin{description}
			\item[Parameter] \hfill \\
			\vspace{-.8cm}
			\begin{itemize}
				\item $\texttt{Observer o}$ \\ Der neue Beobachter.
			\end{itemize}
			\item[Exceptions] \hfill \\
			\vspace{-.8cm}
			\begin{itemize}
				\item $\texttt{NullPointerException}$ \\ Falls $\texttt{o == null}$ ist.
			\end{itemize}
		\end{description}
		
		\item $\texttt{public void \textbf{removeObserver}(Observer o)}$ \\ Entfernt den Beobachter aus der Liste, falls dieser darin existiert, sodass dieser nicht mehr bei Änderungen informiert wird.
		\begin{description}
			\item[Parameter] \hfill \\
			\vspace{-.8cm}
			\begin{itemize}
				\item $\texttt{Observer o}$ \\ Der zu entfernende Beobachter.
			\end{itemize}
			\item[Exceptions] \hfill \\
			\vspace{-.8cm}
			\begin{itemize}
				\item $\texttt{NullPointerException}$ \\ Falls $\texttt{o == null}$ ist.
			\end{itemize}
		\end{description}
		
		\item $\texttt{public void \textbf{notify}(Consumer<Observer> notifier)}$ \\ Ruft die gegebene Funktion auf allen Beobachtern auf. Wird benutzt, um Beobachter über Änderungen am Objekt zu informieren.
		\begin{description}
			\item[Parameter] \hfill \\
			\vspace{-.8cm}
			\begin{itemize}
				\item $\texttt{Consumer<Observer> notifier}$ \\ Die Funktion, die auf allen Beobachtern ausgeführt wird.
			\end{itemize}
			\item[Exceptions] \hfill \\
			\vspace{-.8cm}
			\begin{itemize}
				\item $\texttt{NullPointerException}$ \\ Falls $\texttt{notifier == null}$ ist.
			\end{itemize}
		\end{description}
	\end{itemize}
\end{description}

\subsubsection{\normalfont \texttt{public class \textbf{LambdaGame} extends gdx.Game}}

\begin{description}
\item[Beschreibung] \hfill \\ Stellt die Hauptklasse der Applikation dar.

\item[Attribute] \hfill \\
	\vspace{-.8cm}
	\begin{itemize}
		\item $\texttt{private AchievementMenuViewController \textbf{achievementMenuVC}}$ \\ ViewController zum Achievementmenü
		\item $\texttt{private DropDownMenuViewController \textbf{dropDownMenuVC}}$ \\ ViewController zum Drop-Downmenü
		\item $\texttt{private StatisticViewController \textbf{statisticVC}}$ \\ ViewController zum Statistikmenü
		\item $\texttt{private MainMenuViewController \textbf{mainMenuVC}}$ \\ ViewController zum Hauptmenü
		\item $\texttt{private SettingsViewController \textbf{settingsVC}}$ \\ ViewController zum Einstellungsmenü
		\item $\texttt{private ShopViewController \textbf{shopVC}}$ \\ ViewController zum Shopmenü
		\item $\texttt{private ShopItemViewController \textbf{shopItemVC}}$ \\ ViewController zu einem Shop-Item
		
		\end{itemize}
	
	
\item[Methoden] \hfill \\
	\vspace{-.8cm}
	\begin{itemize}
		\item $\texttt{public AchievementMenuViewController \textbf{getAchievementMenuVC}()}$ \\ Gibt den ViewController zum Achievementmenü zurück
		\begin{description}
			\item[Rückgabe] \hfill \\
			\vspace{-.8cm}
			\begin{itemize}
				\item Gibt $\texttt{achievementMenuVC}$ zurück.
			\end{itemize}
			\end{description}
		
		\item $\texttt{public DropDownMenuViewController \textbf{achievementMenuVC}()}$ \\ Gibt den ViewController zum Drop-Downmenü zurück
		\begin{description}
			\item[Rückgabe] \hfill \\
			\vspace{-.8cm}
			\begin{itemize}
				\item Gibt $\texttt{dropDownMenuVC}$ zurück.
			\end{itemize}
			\end{description}
			
		\item $\texttt{public StatisticViewController \textbf{achievementMenuVC}()}$ \\ Gibt den ViewController zum Statistikmenü zurück
			\begin{description}
			\item[Rückgabe] \hfill \\
			\vspace{-.8cm}
			\begin{itemize}
				\item Gibt $\texttt{statisticVC}$ zurück.
			\end{itemize}
			\end{description}
			
		\item $\texttt{public MainMenuViewController \textbf{achievementMenuVC}()}$ \\ Gibt den ViewController zum Hauptmenü zurück
		\begin{description}
			\item[Rückgabe] \hfill \\
			\vspace{-.8cm}
			\begin{itemize}
				\item Gibt $\texttt{mainMenuVC}$ zurück.
			\end{itemize}
			\end{description}
			
		\item $\texttt{public SettingsViewController \textbf{achievementMenuVC}()}$ \\ Gibt den ViewController zum Einstellungsmenü zurück
		\begin{description}
			\item[Rückgabe] \hfill \\
			\vspace{-.8cm}
			\begin{itemize}
				\item Gibt $\texttt{settingsVC}$ zurück.
			\end{itemize}
			\end{description}
			
		\item $\texttt{public ShopViewController \textbf{achievementMenuVC}()}$ \\ Gibt den ViewController zum Shopmenü zurück
		\begin{description}
			\item[Rückgabe] \hfill \\
			\vspace{-.8cm}
			\begin{itemize}
				\item Gibt $\texttt{shopVC}$ zurück.
			\end{itemize}
			\end{description}
			
		\item $\texttt{public ShopItemViewController \textbf{achievementMenuVC}()}$ \\ Gibt den ViewController zu einem Shop-Item zurück
			\begin{description}
			\item[Rückgabe] \hfill \\
			\vspace{-.8cm}
			\begin{itemize}
				\item Gibt $\texttt{shopItemVC}$ zurück.
			\end{itemize}
			\end{description}

		\item $\texttt{public void \textbf{create}()}$ \\ Erstellt alle ViewController, die in dieser Klasse gehalten werden.
		\item $\texttt{public void \textbf{dispose}()}$ \\ Ruft von jedem ViewController die Methode $\texttt{dispose}$() auf.
		\item $\texttt{public void \textbf{resume}()}$ \\ Ruft von dem aktuell gesetzten ViewController $\texttt{resume}$() auf.
		\item $\texttt{public void \textbf{pause}()}$ \\ Ruft von dem aktuell gesetzten ViewController $\texttt{pause}$() auf.
		\item $\texttt{public void \textbf{render}()}$ \\ Ruft von dem aktuell gesetzten ViewController $\texttt{render}$() auf.
		\item $\texttt{public void \textbf{resize}()}$ \\ Ruft von dem aktuell gesetzten ViewController $\texttt{resize}$() auf.
		

			

		\end{itemize}
	\end{description}
	
	

\subsubsection{\normalfont \texttt{public class \textbf{AssetModel}}}

\begin{description}
\item[Beschreibung] \hfill \\ Enthält alle erforderlichen Daten, welche für das gesamte Spiel benötigt werden. Die Ressourcen werden durch eine JSON-Datei geladen.
\item[Attribute] \hfill \\
	\vspace{-.8cm}
	\begin{itemize}	
		\item $\texttt{private static AssetModel \textbf{assets}}$ \\ Statische Instanz von sich selbst, damit von jeder Klasse auf die Assets zugegriffen werden kann.
		\item $\texttt{private Map<String, Sound> \textbf{sounds}}$ \\ Map, welche alle Sounds enthält, die für das Spiel benötigt werden. Jeder Sound hat einen eindeutigen Bezeichner, welcher als Key dient.
		\item $\texttt{private Map<String, Music> \textbf{music}}$ \\ Map, welche jedes Musikstück enthält, die für das Spiel benötigt werden. Jedes Musikstück hat einen eindeutigen Bezeichner, welcher als Key dient.
		\item $\texttt{private Map<int, DifficultySettings> \textbf{difficultySettings}}$ \\ Map, welche alle Einstellungen für einen Schwierigkeitsgrad eines Levels enthält. Jede Einstellung für einen Schwierigkeitsgrad hat einen eindeutigen Indentfizierer, welcher als Key dient.
		\item $\texttt{private Map<String, Image> \textbf{images}}$ \\ Map, welche alle Bilder enthält, die für das Spiel benötigt werden. Jedes Bild hat einen eindeutigen Bezeichner, welcher als Key dient.
		\item $\texttt{private Map<String, TutorialMessage> \textbf{tutorials}}$ \\ Map, welche alle Anleitungen enthält, die für alle Levels benötigt werden. Jedes Tutorial hat einen eindeutigen Bezeichner, welcher als Key dient.
		\item $\texttt{private Map<int, LevelModel> \textbf{levels}}$ \\ Map, welche alle Level-Modelle enthält, die für das Spiel benötigt werden. Jedes Level-Modell hat einen eindeutigen Identifizierer, welcher als Key dient.

		\end{itemize}
	
\item[Konstruktoren] \hfill \\
	\vspace{-.8cm}
	\begin{itemize}
		\item $\texttt{private \textbf{AssetModel}()}$ \\ instanziiert ein Objekt dieser Klasse.

	\end{itemize}
	
\item[Methoden] \hfill \\
	\vspace{-.8cm}
		\item $\texttt{public AssetModel \textbf{getAssets}()}$ \\ 
		\begin{description}
			\item[Rückgabe] \hfill \\
			\vspace{-.8cm}
			\begin{itemize}
				\item Gibt $\texttt{assets}$ zurück.
			\end{itemize}
			\end{description}
	
		\item $\texttt{public Sound \textbf{getSoundByKey}(String key)}$ \\ 
		\begin{description}
			\item[Parameter] \hfill \\
			\vspace{-.8cm}
			\begin{itemize}
				\item $\texttt{String key}$ \\ Key, um das entsprechende Objekt aus der Map zu holen.
			\end{itemize}
			\item[Rückgabe] \hfill \\
			\vspace{-.8cm}
			\begin{itemize}
				\item Gibt einen $\texttt{Sound}$-Objekt zurück, welches nach dem Parameter $\texttt{key}$ aus der Map $\texttt{sounds}$ geholt wird.
			\end{itemize}
			\end{description}
			
		\item $\texttt{public Music \textbf{getMusicByKey}(String key)}$ \\ 
		\begin{description}
			\item[Parameter] \hfill \\
			\vspace{-.8cm}
			\begin{itemize}
				\item $\texttt{String key}$ \\ Key, um das entsprechende Objekt aus der Map zu holen.
			\end{itemize}
			\item[Rückgabe] \hfill \\
			\vspace{-.8cm}
			\begin{itemize}
				\item Gibt ein $\texttt{Music}$-Objekt zurück, welches nach dem Parameter $\texttt{key}$ aus der Map $\texttt{music}$ geholt wird.
			\end{itemize}
			\end{description}

		\item $\texttt{public DifficultySettings \textbf{getDifficultySettingByKey}(int key)}$ \\ 
		\begin{description}
			\item[Parameter] \hfill \\
			\vspace{-.8cm}
			\begin{itemize}
				\item $\texttt{int key}$ \\ Key, um das entsprechende Objekt aus der Map zu holen.
			\end{itemize}
			\item[Rückgabe] \hfill \\
			\vspace{-.8cm}
			\begin{itemize}
				\item Gibt ein $\texttt{DifficultySetting}$-Objekt zurück, welches nach dem Parameter $\texttt{key}$ aus der Map $\texttt{difficultySettings}$ geholt wird.
			\end{itemize}
			\end{description}
			
		\item $\texttt{public Image \textbf{getImageByKey}(String key)}$\\ 
		\begin{description}
			\item[Parameter] \hfill \\
			\vspace{-.8cm}
			\begin{itemize}
				\item $\texttt{String key}$ \\ Key, um das entsprechende Objekt aus der Map zu holen.
			\end{itemize}
			\item[Rückgabe] \hfill \\
			\vspace{-.8cm}
			\begin{itemize}
				\item Gibt ein $\texttt{Image}$-Objekt zurück, welches nach dem Parameter $\texttt{key}$ aus der Map $\texttt{images}$
			\end{itemize}
			\end{description}

		\item $\texttt{public TutorialMessage \textbf{getTutorialByKey}(String key)}$ \\ 
		\begin{description}
			\item[Parameter] \hfill \\
			\vspace{-.8cm}
			\begin{itemize}
				\item $\texttt{String key}$ \\ Key, um das entsprechende Objekt aus der Map zu holen.
			\end{itemize}
			\item[Rückgabe] \hfill \\
			\vspace{-.8cm}
			\begin{itemize}
				\item Gibt eine $\texttt{TutorialMessage}$-Objekt zurück, welches nach dem Parameter $\texttt{key}$ aus der Map $\texttt{tutorialMessages}$
			\end{itemize}
			\end{description}
			
		\item $\texttt{public LevelModel \textbf{getLevelByKey}(int key)}$ \\ 
		\begin{description}
			\item[Parameter] \hfill \\
			\vspace{-.8cm}
			\begin{itemize}
				\item $\texttt{int key}$ \\ Key, um das entsprechende Objekt aus der Map zu holen.
			\end{itemize}
			\item[Rückgabe] \hfill \\
			\vspace{-.8cm}
			\begin{itemize}
				\item Gibt ein $\texttt{LevelModel}$-Objekt zurück, welches nach dem Parameter $\texttt{key}$ aus der Map $\texttt{levels}$ geholt wird.
			\end{itemize}
			\end{description}
			
	\end{description}

