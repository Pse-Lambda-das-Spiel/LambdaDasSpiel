\subsection{\texttt{package lambda.model.level}}

\subsubsection{\normalfont \texttt{public class \textbf{LevelModel}}}

\begin{description}
\item[Beschreibung] \hfill \\ Repräsentiert ein Level mit allen erforderlichen Daten.

\item[Attribute] \hfill \\
	\vspace{-.8cm}
	\begin{itemize}
		\item $\texttt{private final int \textbf{ID}}$ \\ Ist der eindeutige Identifizierer eines Levels.
		\item $\texttt{private LambdaRoot \textbf{start}}$ \\ Enthält den Startterm eines Levels.
		\item $\texttt{private LambdaRoot \textbf{goal}}$ \\ Enthält den Zielterm eines Levels.
		\item $\texttt{private LambdaRoot \textbf{hint}}$ \\ Enthält einen Lösungshinweis für ein Levels.
		\item $\texttt{private List<TutorialMessage> \textbf{tutorials}}$ \\ Enthält alle Tutorials, die für das Level benötigt werden.
		\item $\texttt{private List<ReductionStrategy> \textbf{availableRedStrats}}$ \\ Enthält alle Reduktionsstrategien, welche man für das Level verwenden darf.
		\item $\texttt{private List<ElementType> \textbf{usableElements}}$ \\ Enthält alle Elementtypen, die im Level in der Elementen-Leiste im Editormodus für das Level verfügbar sind.
		\item $\texttt{private int \textbf{difficulty}}$ \\ Beschreibt den Schwierigkeitsgrad des Levels aufsteigend von 1.
		\item $\texttt{private boolean \textbf{standardMode}}$ \\ Beschreibt, ob man die Anfangs- oder Endkonstellation bestimmen muss, um das Level erfolgreich abzuschließen.
		\end{itemize}
	
\item[Konstruktoren] \hfill \\
	\vspace{-.8cm}
	\begin{itemize}
		\item $\texttt{public \textbf{LevelModel}()}$ \\ instanziiert ein Objekt dieser Klasse.

	\end{itemize}
	
\item[Methoden] \hfill \\
	\vspace{-.8cm}
	\begin{itemize}
		\item $\texttt{public int \textbf{getID}()}$ \\ Gibt die ID des Levels zurück.
		\begin{description}
			\item[Rückgabe] \hfill \\
			\vspace{-.8cm}
			\begin{itemize}
				\item Gibt $\texttt{ID}$ zurück.
			\end{itemize}
			\end{description}
		
		\item $\texttt{public LambdaRoot \textbf{getStart}()}$ \\ Gibt den Starterm des Levels zurück.
		\begin{description}
			\item[Rückgabe] \hfill \\
			\vspace{-.8cm}
			\begin{itemize}
				\item Gibt $\texttt{start}$ zurück.
			\end{itemize}
			\end{description}
			
		\item $\texttt{public LambdaRoot \textbf{getGoal}()}$ \\ Gibt den Zielgerm des Levels zurück.
		\begin{description}
			\item[Rückgabe] \hfill \\
			\vspace{-.8cm}
			\begin{itemize}
				\item Gibt $\texttt{goal}$ zurück.
			\end{itemize}
			\end{description}
			
		\item $\texttt{public LambdaRoot \textbf{getHint}()}$ \\ Gibt einen Hinweis zum Level.
		\begin{description}
			\item[Rückgabe] \hfill \\
			\vspace{-.8cm}
			\begin{itemize}
				\item Gibt $\texttt{hint}$ zurück.
			\end{itemize}
			\end{description}
			
		\item $\texttt{public List<TutorialMessage> \textbf{getTutorials}()}$ \\ Gibt eine Liste mit allen Anleitungen zurück, welche zum Level gehören.
		\begin{description}
			\item[Rückgabe] \hfill \\
			\vspace{-.8cm}
			\begin{itemize}
				\item Gibt $\texttt{tutorials}$ zurück.
			\end{itemize}
			\end{description}
			
		\item $\texttt{public List<ReductionStrategy> \textbf{getAvailableRedStrats}()}$ \\ Gibt eine Liste zurück mit allen Reduktionsstrategien, welche man für dieses Level verwenden darf.
		\begin{description}
			\item[Rückgabe] \hfill \\
			\vspace{-.8cm}
			\begin{itemize}
				\item Gibt $\texttt{availableRedStrats}$ zurück.
			\end{itemize}
			\end{description}
			
		\item $\texttt{public List<ElementType> \textbf{getUsableElements}()}$ \\ Gibt eine Liste mit Elementtypen zurück, welche zum Lösen des Levels benutzt werden dürfen.
			\begin{description}
			\item[Rückgabe] \hfill \\
			\vspace{-.8cm}
			\begin{itemize}
				\item Gibt $\texttt{usableElements}$ zurück.
			\end{itemize}
			\end{description}
			
		\item $\texttt{public int \textbf{getDifficulty}()}$ \\ Gibt den Schwierigkeitsgrad in Form einer Zahl zurück.
		\begin{description}
			\item[Rückgabe] \hfill \\
			\vspace{-.8cm}
			\begin{itemize}
				\item Gibt $\texttt{difficulty}$ zurück.
			\end{itemize}
			\end{description}
			
			\item $\texttt{public boolean \textbf{isStandardMode}()}$ \\ Gibt in Form eines Wahrheitswertes zurück, ob man die Anfangs- oder Entkonstellation bestimmen muss, um das Level erfolgreich abzuschließen.
			\begin{description}
				\item[Rückgabe] \hfill \\
				\vspace{-.8cm}
				\begin{itemize}
					\item Gibt $\texttt{true}$ zurück, falls man die Endkonstellation bestimmen muss, $\texttt{false}$ sonst
				\end{itemize}
				\end{description}

		\end{itemize}
	\end{description}
	
	
	
	

\subsubsection{\normalfont \texttt{public class \textbf{LevelContext}}}

\begin{description}
\item[Beschreibung] \hfill \\ Repräsentiert einen vollständigen Level-Kontext mit allen Daten vom LedelModel, sowie weiteren Angaben zur Hintergrundmusik, Hintergrundbild und der ElementUIContextFamily.

\item[Attribute] \hfill \\
	\vspace{-.8cm}
	\begin{itemize}
		\item $\texttt{private LevelModel \textbf{levelModel}}$ \\ LevelModel, welches alle weiteren Daten enthält, die für den LevelContext ausgelesen werden müssen.
		\item $\texttt{private String \textbf{music}}$ \\ Musik, welche während des Levels im Editor- und Reduktionsmodus im Hintergrund abläuft.
		\item $\texttt{private String \textbf{image}}$ \\ Bild, welches während des Levels im Editor- und Reduktionsmodus im Hintergrund angezeigt wird.
		\item $\texttt{private List<String> \textbf{tutorials}}$ \\ Liste, welche alle Bezeichner für Anleitungen enthält, welche für das Level abgespielt werden müssen.
		\item $\texttt{private ElementUIContextFamily \textbf{elementUIContextFamily}}$ \\ Familie von Sprites, welche im Editormodus die Spielelemente für die Lambda-Abstraktion, die Lambda-Variable und die Lambda-Klammerung darstellen.
		\end{itemize}
	
\item[Konstruktoren] \hfill \\
	\vspace{-.8cm}
	\begin{itemize}
		\item $\texttt{public \textbf{LevelContext}()}$ \\ instanziiert ein Objekt dieser Klasse.

	\end{itemize}
	
\item[Methoden] \hfill \\
	\vspace{-.8cm}
	\begin{itemize}
		\item $\texttt{public LevelModel \textbf{levelModel}}$ \\ Gibt die ID des Levels zurück.
		\begin{description}
			\item[Rückgabe] \hfill \\
			\vspace{-.8cm}
			\begin{itemize}
				\item Gibt $\texttt{levelModel}$ zurück.
			\end{itemize}
			\end{description}
		
		\item $\texttt{public String \textbf{music}}$ \\ Gibt den Starterm des Levels zurück.
		\begin{description}
			\item[Rückgabe] \hfill \\
			\vspace{-.8cm}
			\begin{itemize}
				\item Gibt $\texttt{music}$ zurück.
			\end{itemize}
			\end{description}
			
		\item $\texttt{public String \textbf{image}}$ \\ Gibt den Zielgerm des Levels zurück.
		\begin{description}
			\item[Rückgabe] \hfill \\
			\vspace{-.8cm}
			\begin{itemize}
				\item Gibt $\texttt{image}$ zurück.
			\end{itemize}
			\end{description}
			
		\item $\texttt{public List<String> \textbf{tutorials}}$ \\ Gibt einen Hinweis zum Level.
		\begin{description}
			\item[Rückgabe] \hfill \\
			\vspace{-.8cm}
			\begin{itemize}
				\item Gibt $\texttt{tutorials}$ zurück.
			\end{itemize}
			\end{description}
			
		\item $\texttt{public ElementUIContextFamily \textbf{elementUIContextFamily}}$ \\ Gibt eine Liste mit allen Anleitungen zurück, welche zum Level gehören.
		\begin{description}
			\item[Rückgabe] \hfill \\
			\vspace{-.8cm}
			\begin{itemize}
				\item Gibt $\texttt{elementUIContextFamily}$ zurück.
			\end{itemize}
			\end{description}

		\end{itemize}
	\end{description}
	
	
\subsubsection{\normalfont \texttt{public enum \textbf{ReductionStrategy}}}

\begin{description}
\item[Beschreibung] \hfill \\ Aufzählung von den verschiedenen Reduktionsstrategien. Man benötigt diese Enumeration, um anzugeben welche Strategien zum Lösen des Levels zur Verfügung stehen.

\item[Attribute] \hfill \\
	\vspace{-.8cm}
	\begin{itemize}
		\item $\texttt{\textbf{NORMAL_ORDER}}$ \\ 
		\item $\texttt{\textbf{APPLICATIVE_ORDER}}$ \\ 
		\item $\texttt{\textbf{CALL_BY_NAME}}$ \\ 
		\item $\texttt{\textbf{CALL_BY_VALUE}}$ \\ 
		\end{itemize}
	
\subsubsection{\normalfont \texttt{public enum \textbf{ElementType}}}

\begin{description}
\item[Beschreibung] \hfill \\ Aufzählung von den verschiedenen Elementtypen. Man benötigt diese Enumeration, um anzugeben welche Elementtypen zum Lösen des Levels zur Verfügung stehen.

\item[Attribute] \hfill \\
	\vspace{-.8cm}
	\begin{itemize}
		\item $\texttt{\textbf{VARIABLE}}$ \\ 
		\item $\texttt{\textbf{ABSTRACTION}}$ \\ 
		\item $\texttt{\textbf{PARANTHESIS}}$ \\ 
		\end{itemize}
	
		
	




