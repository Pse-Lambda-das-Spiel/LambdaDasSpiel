\subsection{\texttt{package lambda.model.achievements}}

\subsubsection{\normalfont \texttt{public interface \textbf{AchievementModelObserver}}}

\begin{description}
\item[Beschreibung] \hfill \\ Stellt einen Beobachter eines AchievementModels dar, welcher über Änderungen informiert wird.

\item[Methoden] \hfill \\
	\vspace{-.8cm}
	\begin{itemize}
		\item $\texttt{default public void \textbf{changedLockedState}(int id)}$ \\ Wird aufgerufen um dem Beobachter mitzuteilen, dass sich der Zustand des Erfolgs, also ob dieser freigeschaltet ist oder nicht, geändert hat. Die Standard-Implementierung ist leer.
			\begin{description}
			\item[Parameter] \hfill \\
			\vspace{-.8cm}
			\begin{itemize}
				\item $\texttt{int id}$ \\ Die ID des Erfolgs mit dem geänderten Zustand.
			\end{itemize}
			\end{description}
	\end{itemize}
\end{description}

\subsubsection{\normalfont \texttt{public abstract class \textbf{AchievementModel} implements Observable<AchievementModelObserver>}}

\begin{description}
\item[Beschreibung] \hfill \\ Repräsentiert einen Erfolg.
\item[Attribute] \hfill \\
	\vspace{-.8cm}
	\begin{itemize}
		\item $\texttt{private int \textbf{id}}$ \\ Die kennzeichnende ID des Erfolgs.
		\item $\texttt{private int \textbf{index}}$ \\ Der Index des Erfolgs der dessen Platz in einer geordneten Auflistung bestimmt.
		\item $\texttt{private String \textbf{description}}$ \\ Die Beschreibung des freigeschalteten Erfolgs.
		\item $\texttt{private String \textbf{requirementDescription}}$ \\ Die Beschreibung der Bedingungen des Erfolgs.
		\item $\texttt{private String \textbf{iconPathAchievementLocked}}$ \\Der Pfad zum Piktogramm des nicht freigeschalteten Erfolgs, der beim Laden des jeweiligen Piktogramms hilft.
		\item $\texttt{private String \textbf{iconPathAchievementUnlocked}}$ \\ Der Pfad zum Piktogramm der nicht freigeschalteten Erfolgs, der beim Laden des jeweiligen Piktogramms hilft.
		\item $\texttt{private boolean \textbf{locked}}$ \\ Gibt an, ob die Bedingungen des Erfolgs erfüllt sind und der Erfolgs damit freigeschaltet ist oder nicht.
	\end{itemize}
	
\item[Konstruktoren] \hfill \\
	\vspace{-.8cm}
	\begin{itemize}
		\item $\texttt{public \textbf{AchievementModel}()}$ \\ Instanziiert ein Objekt dieser Klasse. Setzt $\texttt{locked}$ auf $\texttt{false}$.
	\end{itemize}
	
\item[Methoden] \hfill \\
	\vspace{-.8cm}
	\begin{itemize}
		\item $\texttt{public abstract void \textbf{initialize}()}$ \\ Initialisiert den Erfolg.
		
		\item $\texttt{public abstract boolean \textbf{meetRequirements}(StatisticModel statistic)}$ \\ Prüft anhand der übergebenen Statistik, ob die Bedingungen dieses Erfolgs erfüllt sind.
		\begin{description}
		\item[Parameter] \hfill \\
			\vspace{-.8cm}
			\begin{itemize}
				\item $\texttt{StatisticModel statistic}$ \\ Die Statistik anhand derer die Überprüfung geschieht. 
			\end{itemize}
			\item[Rückgabe] \hfill \\
			\vspace{-.8cm}
			\begin{itemize}
				\item Gibt $\texttt{true}$ zurück, falls die Bedingungen dieses Erfolgs erfüllt sind und ansonsten $\texttt{false}$.
			\end{itemize}
			\item[Exceptions] \hfill \\
			\vspace{-.8cm}
			\begin{itemize}
				\item $\texttt{NullPointerException}$ \\ Falls $\texttt{statistic == null}$ ist.
			\end{itemize}
		\end{description}
		
		\item $\texttt{public int \textbf{getId}()}$ \\ Gibt die ID dieses Erfolgs zurück.
		\begin{description}
			\item[Rückgabe] \hfill \\
			\vspace{-.8cm}
			\begin{itemize}
				\item Die ID des Erfolgs.
			\end{itemize}
		\end{description}

\item $\texttt{public int \textbf{getIndex}()}$ \\ Gibt den Index des Erfolgs zurück.
		\begin{description}
			\item[Rückgabe] \hfill \\
			\vspace{-.8cm}
			\begin{itemize}
				\item Der Index des Erfolgs.
			\end{itemize}
		\end{description}

\item $\texttt{public String \textbf{getDescription}()}$ \\ Gibt die Beschreibung für den freigeschalteten Erfolg zurück.
		\begin{description}
			\item[Rückgabe] \hfill \\
			\vspace{-.8cm}
			\begin{itemize}
				\item Die Beschreibung des freigeschalteten Erfolgs.
			\end{itemize}
		\end{description}

\item $\texttt{public String \textbf{getRequirementsDescription}()}$ \\ Gibt die Beschreibung der Bedingungen dieses Erfolgs zurück.
		\begin{description}
			\item[Rückgabe] \hfill \\
			\vspace{-.8cm}
			\begin{itemize}
				\item Die Beschreibung der Bedingungen dieses Erfolgs.
			\end{itemize}
		\end{description}

\item $\texttt{public String \textbf{getIconPathAchievementLocked}()}$ \\ Gibt den Pfad zum Piktogramm des nicht freigeschalteten Erfolgs zurück.
		\begin{description}
			\item[Rückgabe] \hfill \\
			\vspace{-.8cm}
			\begin{itemize}
				\item Der Pfad zum Piktogramm des nicht freigeschalteten Erfolgs.
			\end{itemize}
		\end{description}

\item $\texttt{public String \textbf{getIconPathAchievementUnlocked}()}$ \\ Gibt den Pfad zum Piktogramm des freigeschalteten Erfolgs zurück.
		\begin{description}
			\item[Rückgabe] \hfill \\
			\vspace{-.8cm}
			\begin{itemize}
				\item Der Pfad zum Piktogramm des freigeschalteten Erfolgs.
			\end{itemize}
		\end{description}

\item $\texttt{public boolean \textbf{isLocked}()}$ \\ Gibt an, ob der Erfolg freigeschaltet ist oder nicht.
		\begin{description}
			\item[Rückgabe] \hfill \\
			\vspace{-.8cm}
			\begin{itemize}
				\item Gibt $\texttt{true}$ zurück, falls der Erfolg nicht freigeschaltet ist und $\texttt{false}$ falls der Erfolg freigeschaltet ist.
			\end{itemize}
		\end{description}

	\item $\texttt{public void \textbf{setId}(int id)}$ \\ Setzt eine neue ID für diesen Erfolg.
		\begin{description}
			\item[Parameter] \hfill \\
			\vspace{-.8cm}
			\begin{itemize}
				\item $\texttt{int id}$ \\ Die neue ID.
			\end{itemize}
		\end{description}

	\item $\texttt{public void \textbf{setIndex}(int index)}$ \\ Setzt einen neuen Index für diesen Erfolg
		\begin{description}
			\item[Parameter] \hfill \\
			\vspace{-.8cm}
			\begin{itemize}
				\item $\texttt{int id}$ \\ Der neue Index.
			\end{itemize}
		\end{description}
		
		\item $\texttt{public void \textbf{setDiscription}(String discription)}$ \\ Setzt eine neue Beschreibung für den freigeschalteten Erfolg.
		\begin{description}
			\item[Parameter] \hfill \\
			\vspace{-.8cm}
			\begin{itemize}
				\item $\texttt{String discription}$ \\ Die neue Beschreibung des freigeschalteten Erfolgs.
			\end{itemize}
		\end{description}
		
		\item $\texttt{public void \textbf{setRequirementsDiscription}(String requirementsDescription)}$ \\ Setzt eine neue Beschreibung für die Bedingungen dieses Erfolgs.
		\begin{description}
			\item[Parameter] \hfill \\
			\vspace{-.8cm}
			\begin{itemize}
				\item $\texttt{String requirementsDescription}$ \\ Die neue Beschreibung für die Bedingungen dieses Erfolgs.
			\end{itemize}
		\end{description}

	\item $\texttt{public void \textbf{setIconPathAchievementLocked}(String iconPathAchievementLocked)}$ \\ Setzt einen neuen Pfad zum Piktogramm des nicht freischalteten Erfolgs.
		\begin{description}
			\item[Parameter] \hfill \\
			\vspace{-.8cm}
			\begin{itemize}
				\item $\texttt{String iconPathAchievementLocked}$ \\ Der neue Pfad zum Piktogramm des nicht freigeschalteten Erfolgs.
			\end{itemize}
		\end{description}
	
	\item $\texttt{public void \textbf{setIconPathAchievementUnlocked}(String iconPathAchievementUnlocked)}$ \\ Setzt einen neuen Pfad zum Piktogramm des freischalteten Erfolgs.
		\begin{description}
			\item[Parameter] \hfill \\
			\vspace{-.8cm}
			\begin{itemize}
				\item $\texttt{String iconPathAchievementUnlocked}$ \\ Der neue Pfad zum Piktogramm des freigeschalteten Erfolgs.
			\end{itemize}
		\end{description}
		
	\item $\texttt{public void \textbf{setLocked}(boolean locked)}$ \\ Setzt diesen Erfolg auf den Zustand freigeschaltet oder nicht freigeschaltet.
		\begin{description}
			\item[Parameter] \hfill \\
			\vspace{-.8cm}
			\begin{itemize}
				\item $\texttt{boolean locked}$ \\ Bei $\texttt{true}$ wird der Erfolg auf nicht freigeschaltet und bei $\texttt{false}$ auf freigeschaltet gesetzt.
			\end{itemize}
		\end{description}

	\end{itemize}
\end{description}

\subsubsection{\normalfont \texttt{public class \textbf{TimeAchievementModel} extends AchievementModel}}

\begin{description}
\item[Beschreibung] \hfill \\ Repräsentiert einen Erfolg, der nach dem Spielen einer bestimmten Zeitspanne freigeschaltet wird.
	
\item[Konstruktoren] \hfill \\
	\vspace{-.8cm}
	\begin{itemize}
		\item $\texttt{public \textbf{TimeAchievementModel}()}$ \\ Instanziiert ein Objekt dieser Klasse. Setzt $\texttt{locked}$ auf $\texttt{false}$.
	\end{itemize}
	
\item[Methoden] \hfill \\
	\vspace{-.8cm}
	\begin{itemize}
		\item $\texttt{public void \textbf{initialize}()}$ \\ Initialisiert den Erfolg.
		
		\item $\texttt{public boolean \textbf{meetRequirements}(StatisticModel statistic)}$ \\ Prüft anhand der übergebenen Statistik, ob die Bedingungen dieses Erfolgs erfüllt sind.
		\begin{description}
		\item[Parameter] \hfill \\
			\vspace{-.8cm}
			\begin{itemize}
				\item $\texttt{StatisticModel statistic}$ \\ Die Statistik anhand derer die Überprüfung geschieht. 
			\end{itemize}
			\item[Rückgabe] \hfill \\
			\vspace{-.8cm}
			\begin{itemize}
				\item Gibt $\texttt{true}$ zurück, falls die Bedingungen dieses Erfolgs erfüllt sind und ansonsten $\texttt{false}$.
			\end{itemize}
			\item[Exceptions] \hfill \\
			\vspace{-.8cm}
			\begin{itemize}
				\item $\texttt{NullPointerException}$ \\ Falls $\texttt{statistic == null}$ ist.
			\end{itemize}
		\end{description}
	\end{itemize}
\end{description}

\subsubsection{\normalfont \texttt{public class \textbf{LevelAchievementModel} extends AchievementModel}}

\begin{description}
\item[Beschreibung] \hfill \\ Repräsentiert einen Erfolg, der nach dem erfolgreichen ersten Abschluss einer bestimmten Mindestanzahl von Level freigeschaltet wird.
	
\item[Konstruktoren] \hfill \\
	\vspace{-.8cm}
	\begin{itemize}
		\item $\texttt{public \textbf{LevelAchievementModel}()}$ \\ Instanziiert ein Objekt dieser Klasse. Setzt $\texttt{locked}$ auf $\texttt{false}$.
	\end{itemize}
	
\item[Methoden] \hfill \\
	\vspace{-.8cm}
	\begin{itemize}
		\item $\texttt{public void \textbf{initialize}()}$ \\ Initialisiert den Erfolg.
		
		\item $\texttt{public boolean \textbf{meetRequirements}(StatisticModel statistic)}$ \\ Prüft anhand der übergebenen Statistik, ob die Bedingungen dieses Erfolgs erfüllt sind.
		\begin{description}
		\item[Parameter] \hfill \\
			\vspace{-.8cm}
			\begin{itemize}
				\item $\texttt{StatisticModel statistic}$ \\ Die Statistik anhand derer die Überprüfung geschieht. 
			\end{itemize}
			\item[Rückgabe] \hfill \\
			\vspace{-.8cm}
			\begin{itemize}
				\item Gibt $\texttt{true}$ zurück, falls die Bedingungen dieses Erfolgs erfüllt sind und ansonsten $\texttt{false}$.
			\end{itemize}
			\item[Exceptions] \hfill \\
			\vspace{-.8cm}
			\begin{itemize}
				\item $\texttt{NullPointerException}$ \\ Falls $\texttt{statistic == null}$ ist.
			\end{itemize}
		\end{description}		
		
	\end{itemize}
\end{description}

\subsubsection{\normalfont \texttt{public class \textbf{GemsEnchantedAchievementModel} extends AchievementModel}}

\begin{description}
\item[Beschreibung] \hfill \\ Repräsentiert einen Erfolg, der nach dem Verzaubern einer bestimmten Mindestanzahl von Edelsteinen freigeschaltet wird.
	
\item[Konstruktoren] \hfill \\
	\vspace{-.8cm}
	\begin{itemize}
		\item $\texttt{public \textbf{GemsEnchantedAchievementModel}()}$ \\ Instanziiert ein Objekt dieser Klasse. Setzt $\texttt{locked}$ auf $\texttt{false}$.
	\end{itemize}
	
\item[Methoden] \hfill \\
	\vspace{-.8cm}
	\begin{itemize}
		\item $\texttt{public void \textbf{initialize}()}$ \\ Initialisiert den Erfolg.
		
		\item $\texttt{public boolean \textbf{meetRequirements}(StatisticModel statistic)}$ \\ Prüft anhand der übergebenen Statistik, ob die Bedingungen dieses Erfolgs erfüllt sind.
		\begin{description}
		\item[Parameter] \hfill \\
			\vspace{-.8cm}
			\begin{itemize}
				\item $\texttt{StatisticModel statistic}$ \\ Die Statistik anhand derer die Überprüfung geschieht. 
			\end{itemize}
			\item[Rückgabe] \hfill \\
			\vspace{-.8cm}
			\begin{itemize}
				\item Gibt $\texttt{true}$ zurück, falls die Bedingungen dieses Erfolgs erfüllt sind und ansonsten $\texttt{false}$.
			\end{itemize}
			\item[Exceptions] \hfill \\
			\vspace{-.8cm}
			\begin{itemize}
				\item $\texttt{NullPointerException}$ \\ Falls $\texttt{statistic == null}$ ist.
			\end{itemize}
		\end{description}
		
	\end{itemize}
\end{description}

\subsubsection{\normalfont \texttt{public class \textbf{LambsEnchantedAchievementModel} extends AchievementModel}}

\begin{description}
\item[Beschreibung] \hfill \\ Repräsentiert einen Erfolg, der nach dem Verzaubern einer bestimmten Mindestanzahl von Lämmern freigeschaltet wird.

\item[Konstruktoren] \hfill \\
	\vspace{-.8cm}
	\begin{itemize}
		\item $\texttt{public \textbf{LambsEnchantedAchievementModel}()}$ \\ Instanziiert ein Objekt dieser Klasse. Setzt $\texttt{locked}$ auf $\texttt{false}$.
	\end{itemize}
	
\item[Methoden] \hfill \\
	\vspace{-.8cm}
	\begin{itemize}
				\item $\texttt{public void \textbf{initialize}()}$ \\ Initialisiert den Erfolg.
		
		\item $\texttt{public boolean \textbf{meetRequirements}(StatisticModel statistic)}$ \\ Prüft anhand der übergebenen Statistik, ob die Bedingungen dieses Erfolgs erfüllt sind.
		\begin{description}
		\item[Parameter] \hfill \\
			\vspace{-.8cm}
			\begin{itemize}
				\item $\texttt{StatisticModel statistic}$ \\ Die Statistik anhand derer die Überprüfung geschieht. 
			\end{itemize}
			\item[Rückgabe] \hfill \\
			\vspace{-.8cm}
			\begin{itemize}
				\item Gibt $\texttt{true}$ zurück, falls die Bedingungen dieses Erfolgs erfüllt sind und ansonsten $\texttt{false}$.
			\end{itemize}
			\item[Exceptions] \hfill \\
			\vspace{-.8cm}
			\begin{itemize}
				\item $\texttt{NullPointerException}$ \\ Falls $\texttt{statistic == null}$ ist.
			\end{itemize}
		\end{description}
		
	\end{itemize}
\end{description}

\subsubsection{\normalfont \texttt{public class \textbf{GemsPlacedAchievementModel} extends AchievementModel}}

\begin{description}
\item[Beschreibung] \hfill \\ Repräsentiert einen Erfolg, der nach dem Platzieren einer bestimmten Mindestanzahl von Edelsteinen auf dem Spielfeld freigeschaltet wird.
	
\item[Konstruktoren] \hfill \\
	\vspace{-.8cm}
	\begin{itemize}
		\item $\texttt{public \textbf{GemsPlacedAchievementModel}()}$ \\ Instanziiert ein Objekt dieser Klasse. Setzt $\texttt{locked}$ auf $\texttt{false}$.
	\end{itemize}
	
\item[Methoden] \hfill \\
	\vspace{-.8cm}
	\begin{itemize}
				\item $\texttt{public void \textbf{initialize}()}$ \\ Initialisiert den Erfolg.
		
		\item $\texttt{public boolean \textbf{meetRequirements}(StatisticModel statistic)}$ \\ Prüft anhand der übergebenen Statistik, ob die Bedingungen dieses Erfolgs erfüllt sind.
		\begin{description}
		\item[Parameter] \hfill \\
			\vspace{-.8cm}
			\begin{itemize}
				\item $\texttt{StatisticModel statistic}$ \\ Die Statistik anhand derer die Überprüfung geschieht. 
			\end{itemize}
			\item[Rückgabe] \hfill \\
			\vspace{-.8cm}
			\begin{itemize}
				\item Gibt $\texttt{true}$ zurück, falls die Bedingungen dieses Erfolgs erfüllt sind und ansonsten $\texttt{false}$.
			\end{itemize}
			\item[Exceptions] \hfill \\
			\vspace{-.8cm}
			\begin{itemize}
				\item $\texttt{NullPointerException}$ \\ Falls $\texttt{statistic == null}$ ist.
			\end{itemize}
		\end{description}
		
	\end{itemize}
\end{description}

\subsubsection{\normalfont \texttt{public class \textbf{LambsPlacedAchievementModel} extends AchievementModel}}

\begin{description}
\item[Beschreibung] \hfill \\ Repräsentiert einen Erfolg, der nach dem Platzieren einer bestimmten Mindestanzahl von Lämmern auf dem Spielfeld freigeschaltet wird.
	
\item[Konstruktoren] \hfill \\
	\vspace{-.8cm}
	\begin{itemize}
		\item $\texttt{public \textbf{LambsPlacedAchievementModel}()}$ \\ Instanziiert ein Objekt dieser Klasse. Setzt $\texttt{locked}$ auf $\texttt{false}$.
	\end{itemize}
	
\item[Methoden] \hfill \\
	\vspace{-.8cm}
	\begin{itemize}
		\item $\texttt{public void \textbf{initialize}()}$ \\ Initialisiert den Erfolg.
		
		\item $\texttt{public boolean \textbf{meetRequirements}(StatisticModel statistic)}$ \\ Prüft anhand der übergebenen Statistik, ob die Bedingungen dieses Erfolgs erfüllt sind.
		\begin{description}
		\item[Parameter] \hfill \\
			\vspace{-.8cm}
			\begin{itemize}
				\item $\texttt{StatisticModel statistic}$ \\ Die Statistik anhand derer die Überprüfung geschieht. 
			\end{itemize}
			\item[Rückgabe] \hfill \\
			\vspace{-.8cm}
			\begin{itemize}
				\item Gibt $\texttt{true}$ zurück, falls die Bedingungen dieses Erfolgs erfüllt sind und ansonsten $\texttt{false}$.
			\end{itemize}
			\item[Exceptions] \hfill \\
			\vspace{-.8cm}
			\begin{itemize}
				\item $\texttt{NullPointerException}$ \\ Falls $\texttt{statistic == null}$ ist.
			\end{itemize}
		\end{description}
		
	\end{itemize}
\end{description}

\subsubsection{\normalfont \texttt{public class \textbf{HintsAchievementModel} extends AchievementModel}}

\begin{description}
\item[Beschreibung] \hfill \\ Repräsentiert einen Erfolg, der freigeschaltet wird, nachdem eine bestimmte Mindestanzahl von Level erfolgreich ohne Nutzung des Hinweises abgeschlossen wurden.
	
\item[Konstruktoren] \hfill \\
	\vspace{-.8cm}
	\begin{itemize}
		\item $\texttt{public \textbf{HintslAchievementModel}()}$ \\ Instanziiert ein Objekt dieser Klasse. Setzt $\texttt{locked}$ auf $\texttt{false}$.
	\end{itemize}
	
\item[Methoden] \hfill \\
	\vspace{-.8cm}
	\begin{itemize}
		\item $\texttt{public void \textbf{initialize}()}$ \\ Initialisiert den Erfolg.
		
		\item $\texttt{public boolean \textbf{meetRequirements}(StatisticModel statistic)}$ \\ Prüft anhand der übergebenen Statistik, ob die Bedingungen dieses Erfolgs erfüllt sind.
		\begin{description}
		\item[Parameter] \hfill \\
			\vspace{-.8cm}
			\begin{itemize}
				\item $\texttt{StatisticModel statistic}$ \\ Die Statistik anhand derer die Überprüfung geschieht. 
			\end{itemize}
			\item[Rückgabe] \hfill \\
			\vspace{-.8cm}
			\begin{itemize}
				\item Gibt $\texttt{true}$ zurück, falls die Bedingungen dieses Erfolgs erfüllt sind und ansonsten $\texttt{false}$.
			\end{itemize}
			\item[Exceptions] \hfill \\
			\vspace{-.8cm}
			\begin{itemize}
				\item $\texttt{NullPointerException}$ \\ Falls $\texttt{statistic == null}$ ist.
			\end{itemize}
		\end{description}
		
	\end{itemize}
\end{description}

\subsubsection{\normalfont \texttt{public abstract class \textbf{PerLevelAchievementModel} extends AchievementModel}}

\begin{description}
\item[Beschreibung] \hfill \\ Repräsentiert einen Erfolg, dessen Freischaltung eine bestimmte Mindestanzahl von Ereignissen in einem Level erfordert.

\item[Konstruktoren] \hfill \\
	\vspace{-.8cm}
	\begin{itemize}
		\item $\texttt{public \textbf{PerLevelAchievementModel}()}$ \\ Instanziiert ein Objekt dieser Klasse. Setzt $\texttt{locked}$ auf $\texttt{false}$.
	\end{itemize}

\end{description}



\subsubsection{\normalfont \texttt{public class \textbf{GemsEnchantedPerLevelAchievementModel} extends PerLevelAchievementModel}}

\begin{description}
\item[Beschreibung] \hfill \\ Repräsentiert einen Erfolg, der freigeschaltet wird, falls eine bestimmte Mindestanzahl von Edelsteinen in einem Level verzaubert werden.
	
\item[Konstruktoren] \hfill \\
	\vspace{-.8cm}
	\begin{itemize}
		\item $\texttt{public \textbf{GemsEnchantedPerLevelAchievementModel}()}$ \\ Instanziiert ein Objekt dieser Klasse. Setzt $\texttt{locked}$ auf $\texttt{false}$.
	\end{itemize}
	
\item[Methoden] \hfill \\
	\vspace{-.8cm}
	\begin{itemize}
				\item $\texttt{public void \textbf{initialize}()}$ \\ Initialisiert den Erfolg.
		
		\item $\texttt{public boolean \textbf{meetRequirements}(StatisticModel statistic)}$ \\ Prüft anhand der übergebenen Statistik, ob die Bedingungen dieses Erfolgs erfüllt sind.
		\begin{description}
		\item[Parameter] \hfill \\
			\vspace{-.8cm}
			\begin{itemize}
				\item $\texttt{StatisticModel statistic}$ \\ Die Statistik anhand derer die Überprüfung geschieht. 
			\end{itemize}
			\item[Rückgabe] \hfill \\
			\vspace{-.8cm}
			\begin{itemize}
				\item Gibt $\texttt{true}$ zurück, falls die Bedingungen dieses Erfolgs erfüllt sind und ansonsten $\texttt{false}$.
			\end{itemize}
			\item[Exceptions] \hfill \\
			\vspace{-.8cm}
			\begin{itemize}
				\item $\texttt{NullPointerException}$ \\ Falls $\texttt{statistic == null}$ ist.
			\end{itemize}
		\end{description}
	\end{itemize}
\end{description}

\subsubsection{\normalfont \texttt{public class \textbf{LambsEnchantedPerLevelAchievementModel} extends PerLevelAchievementModel}}

\begin{description}
\item[Beschreibung] \hfill \\ Repräsentiert einen Erfolg, der freigeschaltet wird, wenn eine bestimmte Mindestanzahl Lämmern in einem Level verzaubert werden.
	
\item[Konstruktoren] \hfill \\
	\vspace{-.8cm}
	\begin{itemize}
		\item $\texttt{public \textbf{LambsEnchantedPerLevelAchievementModel}()}$ \\ Instanziiert ein Objekt dieser Klasse. Setzt $\texttt{locked}$ auf $\texttt{false}$.
	\end{itemize}
	
\item[Methoden] \hfill \\
	\vspace{-.8cm}
	\begin{itemize}
				\item $\texttt{public void \textbf{initialize}()}$ \\ Initialisiert den Erfolg
		
		\item $\texttt{public boolean \textbf{meetRequirements}(StatisticModel statistic)}$ \\ Prüft anhand der übergebenen Statistik, ob die Bedingungen dieses Erfolgs erfüllt sind.
		\begin{description}
		\item[Parameter] \hfill \\
			\vspace{-.8cm}
			\begin{itemize}
				\item $\texttt{StatisticModel statistic}$ \\ Die Statistik anhand derer die Überprüfung geschieht. 
			\end{itemize}
			\item[Rückgabe] \hfill \\
			\vspace{-.8cm}
			\begin{itemize}
				\item Gibt $\texttt{true}$ zurück, falls die Bedingungen dieses Erfolgs erfüllt sind und ansonsten $\texttt{false}$.
			\end{itemize}
			\item[Exceptions] \hfill \\
			\vspace{-.8cm}
			\begin{itemize}
				\item $\texttt{NullPointerException}$ \\ Falls $\texttt{statistic == null}$ ist.
			\end{itemize}
		\end{description}
	\end{itemize}
\end{description}

\subsubsection{\normalfont \texttt{public class \textbf{GemsPlacedPerLevelAchievementModel} extends PerLevelAchievementModel}}

\begin{description}
\item[Beschreibung] \hfill \\ Repräsentiert einen Erfolg, der freigeschaltet wird, falls eine bestimmte Mindestanzahl von Edelsteinen in einem Level auf dem Spielfeld platziert wird.
	
\item[Konstruktoren] \hfill \\
	\vspace{-.8cm}
	\begin{itemize}
		\item $\texttt{public \textbf{GemsPlacedPerLevelAchievementModel}()}$ \\ Instanziiert ein Objekt dieser Klasse. Setzt $\texttt{locked}$ auf $\texttt{false}$.
	\end{itemize}
	
\item[Methoden] \hfill \\
	\vspace{-.8cm}
	\begin{itemize}
				\item $\texttt{public void \textbf{initialize}()}$ \\ Initialisiert den Erfolg.
		
		\item $\texttt{public boolean \textbf{meetRequirements}(StatisticModel statistic)}$ \\ Prüft anhand der übergebenen Statistik, ob die Bedingungen dieses Erfolgs erfüllt sind.
		\begin{description}
		\item[Parameter] \hfill \\
			\vspace{-.8cm}
			\begin{itemize}
				\item $\texttt{StatisticModel statistic}$ \\ Die Statistik anhand derer die Überprüfung geschieht. 
			\end{itemize}
			\item[Rückgabe] \hfill \\
			\vspace{-.8cm}
			\begin{itemize}
				\item Gibt $\texttt{true}$ zurück, falls die Bedingungen dieses Erfolgs erfüllt sind und ansonsten $\texttt{false}$.
			\end{itemize}
			\item[Exceptions] \hfill \\
			\vspace{-.8cm}
			\begin{itemize}
				\item $\texttt{NullPointerException}$ \\ Falls $\texttt{statistic == null}$ ist.
			\end{itemize}
		\end{description}
	\end{itemize}
\end{description}

\subsubsection{\normalfont \texttt{public class \textbf{LambsPlacedPerLevelAchievementModel} extends PerLevelAchievementModel}}

\begin{description}
\item[Beschreibung] \hfill \\ Repräsentiert einen Erfolg, der freigeschaltet wird, wenn eine bestimmte Mindestanzahl von Lämmern in einem Level auf dem Spielfeld platziert wird.
	
\item[Konstruktoren] \hfill \\
	\vspace{-.8cm}
	\begin{itemize}
		\item $\texttt{public \textbf{LambsPlacedPerLevelAchievementModel}()}$ \\ Instanziiert ein Objekt dieser Klasse. Setzt $\texttt{locked}$ auf $\texttt{false}$.
	\end{itemize}
	
\item[Methoden] \hfill \\
	\vspace{-.8cm}
	\begin{itemize}
				\item $\texttt{public void \textbf{initialize}()}$ \\ Initialisiert den Erfolg.
		
		\item $\texttt{public boolean \textbf{meetRequirements}(StatisticModel statistic)}$ \\ Prüft anhand der übergebenen Statistik, ob die Bedingungen dieses Erfolgs erfüllt sind.
		\begin{description}
		\item[Parameter] \hfill \\
			\vspace{-.8cm}
			\begin{itemize}
				\item $\texttt{StatisticModel statistic}$ \\ Die Statistik anhand derer die Überprüfung geschieht. 
			\end{itemize}
			\item[Rückgabe] \hfill \\
			\vspace{-.8cm}
			\begin{itemize}
				\item Gibt $\texttt{true}$ zurück, falls die Bedingungen dieses Erfolgs erfüllt sind und ansonsten $\texttt{false}$.
			\end{itemize}
			\item[Exceptions] \hfill \\
			\vspace{-.8cm}
			\begin{itemize}
				\item $\texttt{NullPointerException}$ \\ Falls $\texttt{statistic == null}$ ist.
			\end{itemize}
		\end{description}
	\end{itemize}
\end{description}

\subsubsection{\normalfont \texttt{public class \textbf{AchievementManager}}}

\begin{description}
\item[Beschreibung] \hfill \\ Diese Klasse dient zur Verwaltung aller Erfolge.
\item[Attribute] \hfill \\
	\vspace{-.8cm}
	\begin{itemize}
		\item $\texttt{private Map<int, AchievementModel> \textbf{unlockedAchievements}}$ \\ Ansammlung aller freigeschalteten Erfolge.
		\item $\texttt{private Map<int, AchievementModel> \textbf{lockedAchievements}}$ \\ Ansammlung aller nicht freigeschalteten Erfolge.
	\end{itemize}
	
\item[Konstruktoren] \hfill \\
	\vspace{-.8cm}
	\begin{itemize}
		\item $\texttt{public \textbf{AchievementManager}()}$ \\ Instanziiert ein Objekt dieser Klasse.
	\end{itemize}
	
\item[Methoden] \hfill \\
	\vspace{-.8cm}
	\begin{itemize}
	
		\item $\texttt{public void \textbf{initializeAchievements}()}$ \\ Initialisiert alle Erfolge.
	
		\item $\texttt{public void \textbf{checkUnlockedAchievements}(StatisticModel statistic)}$ \\ Überprüft alle nicht freigeschalteten Erfolge anhand der übergebenen und aktuellen Statistik, ob deren Bedingungen erfüllt sind und sortiert sie gegebenenfalls in die Ansammlung  der freigeschalteten Erfolge ein.
		\begin{description}
		\item[Parameter] \hfill \\
			\vspace{-.8cm}
			\begin{itemize}
				\item $\texttt{StatisticModel statistic}$ \\ Die Statistik anhand derer die Überprüfung aller nicht freigeschalteter Erfolge geschieht. 
			\end{itemize}
			\item[Exceptions] \hfill \\
			\vspace{-.8cm}
			\begin{itemize}
				\item $\texttt{NullPointerException}$ \\ Falls $\texttt{statistic == null}$ ist.
			\end{itemize}
		\end{description}	
		
		\item $\texttt{public Map<int, AchievementModel> \textbf{getUnlockedAchievements}()}$ \\ Gibt die Ansammlung aller freigeschalteter Erfolge zurück.
		\begin{description}
			\item[Rückgabe] \hfill \\
			\vspace{-.8cm}
			\begin{itemize}
				\item Die Ansammlung aller freigeschalteter Erfolge.
			\end{itemize}
		\end{description}
		
		\item $\texttt{public Map<int, AchievementModel> \textbf{getLockedAchievements}()}$ \\ Gibt die Ansammlung aller nicht freigeschalteter Erfolge zurück.
		\begin{description}
			\item[Rückgabe] \hfill \\
			\vspace{-.8cm}
			\begin{itemize}
				\item Die Ansammlung aller nicht freigeschalteten Erfolge.
			\end{itemize}
		\end{description}
		
		\item $\texttt{public boolean \textbf{addAchievement}(AchievementModel achievement)}$ \\ Überprüft, ob der Erfolg freigeschaltet ist oder nicht und fügt ihn anschließend der entsprechenden Ansammlung hinzu.
		\begin{description}
			\item[Parameter] \hfill \\
			\vspace{-.8cm}
			\begin{itemize}
				\item $\texttt{AchievementModel achievement}$ \\ Der zu hinzufügende Erfolg.
			\end{itemize}
			\item[Rückgabe] \hfill \\
			\vspace{-.8cm}
			\begin{itemize}
				\item Gibt $\texttt{true}$ zurück,falls das Hinzufügen in einer der Ansammlungen erfolgreich war oder $\texttt{false}$, falls der Erfolg schon in einer der Ansammlungen vorhanden ist und das Hinzufügen damit nicht erfolgreich war.
			\end{itemize}
			\item[Exceptions] \hfill \\
			\vspace{-.8cm}
			\begin{itemize}
				\item $\texttt{NullPointerException}$ \\ Falls $\texttt{achievement == null}$ ist.
			\end{itemize}
		\end{description}	

	\item $\texttt{public boolean \textbf{moveToUnlockedAchievements}(int id)}$ \\ Verschiebt den bereits vorhandenen Erfolg von der Ansammlung der nicht freigeschalteten Erfolge zu der Ansammlung der freigeschalteten Erfolge.
		\begin{description}
			\item[Parameter] \hfill \\
			\vspace{-.8cm}
			\begin{itemize}
				\item $\texttt{int id}$ \\ Die ID des zu verschiebenden Erfolgs.
			\end{itemize}
			\item[Rückgabe] \hfill \\
			\vspace{-.8cm}
			\begin{itemize}
				\item Gibt $\texttt{true}$ zurück,falls das Verschieben des Erfolgs erfolgreich war und $\texttt{false}$, falls sich der Erfolg bereits in der Ansammlung der freigeschalteten Erfolge befand und damit das Verschieben nicht erfolgreich war.
			\end{itemize}
			\item[Exceptions] \hfill \\
			\vspace{-.8cm}
			\begin{itemize}
				\item $\texttt{IllegalArgumentException}$ \\ Falls $\texttt{id}$ in keiner der beiden Ansammlungen von Erfolgen vorhanden ist.
			\end{itemize}
		\end{description}	

\item $\texttt{public boolean \textbf{moveToLockedAchievements}(int id)}$ \\ Verschiebt die bereits vorhandenen Erfolge von der Ansammlung der freigeschalteten Erfolge zu der Ansammlung der nicht freigeschalteten Erfolge.
		\begin{description}
			\item[Parameter] \hfill \\
			\vspace{-.8cm}
			\begin{itemize}
				\item $\texttt{int id}$ \\ Die ID des zu verschiebenden Erfolgs.
			\end{itemize}
			\item[Rückgabe] \hfill \\
			\vspace{-.8cm}
			\begin{itemize}
				\item Gibt $\texttt{true}$ zurück,falls das Verschieben des Erfolgs erfolgreich war und $\texttt{false}$, falls sich der Erfolg bereits in der Ansammlung der nicht freigeschalteten Erfolge befand und damit das Verschieben nicht erfolgreich war.
			\end{itemize}
			\item[Exceptions] \hfill \\
			\vspace{-.8cm}
			\begin{itemize}
				\item $\texttt{IllegalArgumentException}$ \\ Falls $\texttt{id}$ in keiner der beiden Ansammlungen von Erfolgen vorhanden ist.
			\end{itemize}
		\end{description}	
	\end{itemize}
\end{description}