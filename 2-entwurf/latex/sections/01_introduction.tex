\section{Einleitung}

Die Applikation "`Lambd.da"' soll Kindern im Grundschulalter auf eine spielerische Art und Weise die wesentlichen Aspekte des untypisierten Lambda-Kalküls und damit auch die Grundlage der funktionalen Programmierung vermitteln. In unserer Entwurfsdokumentation beschreiben und modellieren wir unsere Entwurfsentscheidungen und präsentieren dabei auch die Softwarearchitektur unserer Applikation.

Zunächst beschreiben wir die Funktionen der Applikation, die sich während der Entwurfsphase erst herauskristallisiert haben und deshalb noch nicht im Pflichtenheft erwähnt wurden. Anschließend wird erwähnt, welche der im Pflichtenheft genannten Wunschkriterien nicht mehr umgesetzt werden, da sich bereits in der Entwurfsphase ergab, dass wir diese nicht umsetzen können aus verschiedenen Gründen.

Im Kapitel Grobentwurf erläutern wir dann die von uns gewählten Designentscheidungen wie beispielsweise die eingesetzten Entwurfsmusters und beschreiben die Grobstruktur unserer Klassenpakete. Der Hauptteil und dabei auch der umfassendste Teil unseres Entwurfsdokuments bildet jedoch das Kapitel Feinentwurf mit unserer Klassendokumentation, in der alle Klassen und deren Methoden sowie deren Attribute und mögliche auftretende Exceptions aufgelistet und beschrieben werden. Es werden auch unsere eigenen verwendeten Interfaces beschrieben. Passend dazu fügen wir noch UML-Klassendiagramme zu diesem Entwurfsdokument an, in denen unsere beschriebenen Klassen und deren Komponenten als auch die Beziehungen zwischen den Klassen modelliert werden.

Des Weiteren Erläutern wir im Kapitel Datenstrukturen, wie die dauerhaft zu speichernden, logischen Komponenten unserer Applikation gespeichert und verwaltet und unsere Assets wie die Level des Spiels geladen werden. Wichtige Programmabläufe und die daraus resultierende Interaktion der Klassen untereinander werden durch UML-Sequenzdiagramme im Kapitel dynamische Diagramme beschrieben.