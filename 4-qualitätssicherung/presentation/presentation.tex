%% LaTeX-Beamer template for KIT design
%% by Erik Burger, Christian Hammer
%% title picture by Klaus Krogmann
%%
%% version 2.1
%%
%% mostly compatible to KIT corporate design v2.0
%% http://intranet.kit.edu/gestaltungsrichtlinien.php
%%
%% Problems, bugs and comments to
%% burger@kit.edu
\pdfminorversion=4

\documentclass[18pt]{beamer}

\usepackage[utf8]{inputenc}
\usepackage{listings}
%% SLIDE FORMAT

% use 'beamerthemekit' for standard 4:3 ratio
% for widescreen slides (16:9), use 'beamerthemekitwide'

\usepackage{templates/beamerthemekit}
% \usepackage{templates/beamerthemekitwide}

%% TITLE PICTURE

% if a custom picture is to be used on the title page, copy it into the 'logos'
% directory, in the line below, replace 'mypicture' with the 
% filename (without extension) and uncomment the following line
% (picture proportions: 63 : 20 for standard, 169 : 40 for wide
% *.eps format if you use latex+dvips+ps2pdf, 
% *.jpg/*.png/*.pdf if you use pdflatex)

\titleimage{../Media/background}

%% TITLE LOGO

% for a custom logo on the front page, copy your file into the 'logos'
% directory, insert the filename in the line below and uncomment it

\titlelogo{logo}

% (*.eps format if you use latex+dvips+ps2pdf,
% *.jpg/*.png/*.pdf if you use pdflatex)

%% TikZ INTEGRATION

% use these packages for PCM symbols and UML classes
% \usepackage{templates/tikzkit}
% \usepackage{templates/tikzuml}

\usepackage{hyperref}
\usepackage{color}

% speech bubbles
\usepackage{tikz}
\usetikzlibrary{shapes.callouts}
\usetikzlibrary{arrows,positioning}
\usetikzlibrary{calc}

% rotations
\usepackage{rotating}

% strike through in math mode
\usepackage{cancel}

% for embedded video
\usepackage{multimedia}

% for multiline comments
\usepackage{verbatim} 

\usepackage{color}

% the presentation starts here

\title[Lambda Spiel]{Lamb.da - Das Spiel: Qualitätssicherung}
%\subtitle{Spielend die Denkweise des Programmierens lernen}
\author{PSE 2014/2015}

\institute{Farid Elhaddad | Florian Fervers | Kai Fieger | Robert Hochweiss | Kay Schmitteckert}
\date{\today}

\beamertemplatenavigationsymbolsempty

\begin{document}

% change the following line to "ngerman" for German style date and logos
\selectlanguage{ngerman}

\begin{frame}
	\titlepage
\end{frame}

\begin{frame}
	\frametitle{Änderung $\alpha$-Konversion}
	\includegraphics<1>[width=\textwidth]{pictures/conversion1}
	\includegraphics<2>[width=\textwidth]{pictures/conversion2}
	\includegraphics<3>[width=\textwidth]{pictures/conversion3}
\end{frame}

\begin{frame}
	\frametitle{Verbesserungen}
	\begin{itemize}[<+->]
		\item Animationen
		\item Verbesserte Benutzeroberfläche
		\item Mehr Shop-Items
		\item Unterschiedliche Musik und Hintergründe
		\item Level und Tutorials komplett überarbeitet
	\end{itemize}
\end{frame}

\begin{frame}
	\frametitle{Verbesserungen}
	\includegraphics<1>[width=\textwidth]{pictures/tutorial1}
	\includegraphics<2>[width=\textwidth]{pictures/tutorial2}
	\includegraphics<3>[width=\textwidth]{pictures/tutorial3}
	\includegraphics<4>[width=\textwidth]{pictures/tutorial4}
	\includegraphics<5>[width=\textwidth]{pictures/tutorial5}
\end{frame}

\begin{frame}
	\frametitle{Bugs}
	\includegraphics<1>[width=\textwidth]{pictures/bug1}
	\includegraphics<2>[width=\textwidth]{pictures/bug2}
\end{frame}

\begin{frame}
	\frametitle{Bugs}
	\begin{itemize}[<+->]
		\item Verschwindende Klammern
		\item abgeschnittene Levelziele/Hinweise
		\item Schreibfehler in einer Json
		\item ...
		\item viele kleine Bugs
	\end{itemize}
\end{frame}

\begin{frame}
	\frametitle{Testwerkzeuge}
	\begin{itemize}[<+->]
		\item statische Werkzeuge
	\begin{itemize}
		\item Checkstyle
		\item FindBugs
	\end{itemize}
		\item dynamische Werkzeuge
	\begin{itemize}
		\item JUnit
		\item EMMA
		\item MonkeyRunner
		\item Monkey
	\end{itemize}
	\end{itemize}
\end{frame}

\begin{frame}
	\frametitle{Unit-Tests}
	\includegraphics<1>[width=\textwidth]{pictures/coverage1}
	\includegraphics<2>[width=\textwidth]{pictures/coverage2}
\end{frame}

\begin{frame}
	\frametitle{Statistiken}
	Für die Applikation:
	\begin{itemize}
		\item 8764 $\Rightarrow$ 10890 Lines of Code
		\item 26 Packages
		\item 13 Interfaces
		\item 134 $\Rightarrow$ 160 Klassen
		\begin{itemize}
			\item 348 $\Rightarrow$ 401 Attribute
			\item 733 $\Rightarrow$ 853 Methoden
		\end{itemize}
	\end{itemize}
	Für die Tests:
	\begin{itemize}
		\item 752 $\Rightarrow$ 1614 Lines of Code
	\end{itemize}
\end{frame}

\begin{frame}
	\centering
	\huge Vielen Dank für Ihre Aufmerksamkeit!
	\includegraphics[scale=0.8]{team_lambda_lg.png}
\end{frame}

\end{document}
